\section{Avtomorfizmi Riemannovih poloskev}

\subsection{Sfere in torusi}

Za določanje grupe avtomorfizmov Riemannovih ploskev so pomembne
njihove topološke lastnosti -- vsak avtomorfizem je namreč tudi
homeomorfizem. Iz geometrijske topologije vemo, da je vsaka
orientabilna kompaktna ploskev homeomorfna vsoti $g$ torusov.
Številu $g$ pravimo rod ploskve.

Najprej si oglejmo ploskve z ničelnim rodom -- topološko so to kar
sfere. V prejšnjih razdelkih smo ugotovili, da je grupa
avtomorfizmov Riemannove sfere enaka
\[
\Aut \br{\rs} = \setb{\frac{az + b}{cz + d}}{ad - bc = 1}.
\]

Vemo pa, da je grupa avtomorfizmov odvisna ne samo od topoloških
lastnosti objekta, ampak tudi njegove kompleksne strukture. Izkaže
se, da imamo na ploskvah roda $g = 0$ do konformne ekvivalence le
eno kompleksno strukturo:

\begin{lema}
Naj bo $M$ kompaktna Riemannova ploskev roda $g = 0$. Tedaj je $M$
konformno ekvivalentna Riemannovi sferi.
\end{lema}

\begin{proof}
Naj bo $P \in M$ poljubna točka. Po Riemann-Rochovem izreku velja
\[
r \br{P^{-1}} = 1 - 0 + 1 + i(P) \geq 2,
\]
zato obstaja nekonstantna meromorfna funkcija $f \in L \br{P^-1}$.
Ker ima ta pol, je ta v točki $P$ in je enostaven. Sledi, da je
stopnja preslikave $f$ enaka $1$, zato je $f$ bijektivna.
\end{proof}

Posledica zgornje leme je, da imajo vse Riemannove ploskve roda
$g = 0$ izomorfne grupe avtomorfizmov. Lema je pravzaprav poseben
primer Koebejevega izreka o uniformizaciji, ki pravi, da je vsaka
enostavno povezana Riemannova ploskev konformno ekvivalenta
Riemannovi sferi, kompleksni ravnini ali enotskemu disku.

Naslednji izziv so ploskve z rodom $g=1$ -- torusi. Za toruse
izrek o uniformizaciji seveda ne velja, zato bomo dobili drugačne
(in tudi med seboj različne) grupe avtomorfizmov.

Topološko je torus kvocient $\kvoc{\R^2}{\Z^2}$, pri čemer $\Z^2$
deluje na $\R^2$ s predpisom $(m,n) \cdot (x,y) = (x+m, y+n)$.
Množico $\R^2$, ki ji pravimo tudi \emph{krovni prostor}, lahko
seveda enačimo s $\C$. Kvocient $\kvoc{\C}{\Z^2}$ je spet
Riemannova ploskev s podedovano kompleksno
strukturo~\cite[razdelek~1.8.2]{Forstneric}.

Seveda pa lahko za delovanje grupe $\Z^2$ vzamemo tudi kak drug
predpis, na primer $(m,n) \cdot z = z + m \lambda+ n \mu$ za
$\R$-linearno neodvisna $\lambda, \mu \in \C$. Topološko spet
dobimo enak prostor ne glede na izbiro
$\lambda, \mu \in \C \setminus \R$, izkaže pa se, da tako dobimo
bistveno različne kompleksne strukture. To lahko dokažemo tako, da
poiščemo grupe avtomorfizmov.

Naj bo $\Lambda = \setb{m \lambda + n \mu}{m, n \in \Z}$, torej
neka mreža v ravnini. Kompleksen torus, generiran s to mrežo,
označimo s $\kvoc{\C}{\Lambda}$.

Najprej si oglejmo primere preprostih avtomorfizmov, ki jih najdemo
na vsakem torusu. Najpreprostejša geometrijska transformacija
torusa je kar rotacija -- v krovnem prostoru to ustreza preslikavi
$z \mapsto z + r \cdot \lambda$ za realno število $r$. Obstaja pa
še ena vrsta rotacije, pri kateri se torus rotira ">sam vase"<.
To so še preslikave $z \mapsto z + r \cdot \mu$. Skupaj tako dobimo
množico preslikav $z \mapsto z + \alpha$ za poljuben
$\alpha \in \C$.

\begin{figure}[!ht]
\centering
\begin{asy}
void center(pair O) {
	guide lower = O+(-0.8,-0.05)..O+(0,-0.15)..O+(0.8,-0.05);
	guide upper = O+(-0.8,-0.05)..O+(0,0.2)..O+(0.8,-0.05);
	draw(upper);
	draw(O+(-1.1,0.05)..lower..O+(1.1,0.05));
}

draw((-2,0)..(0,1)..(2,0)..(0,-1)..cycle);
//draw((0,-0.9)--(0,-0.25), arrow=Arrow);

draw((0.5,-0.85)..(0.475,-0.3)..(0.45,-0.2), arrow=Arrow);
draw((1.4,-0.7)..(1.3,-0.1)..(1.2,0), arrow=Arrow);
draw((1.5,0.6)..(1.2,0.55)..(0.8,0.15), arrow=Arrow);
draw((0.5,0.85)..(0.42,0.7)..(0.3,0.25), arrow=Arrow);

draw((-0.5,-0.85)..(-0.475,-0.3)..(-0.45,-0.2), arrow=Arrow);
draw((-1.4,-0.7)..(-1.3,-0.1)..(-1.2,0), arrow=Arrow);
draw((-1.5,0.6)..(-1.2,0.55)..(-0.8,0.15), arrow=Arrow);
draw((-0.5,0.85)..(-0.42,0.7)..(-0.3,0.25), arrow=Arrow);
center(0);
\end{asy}
\caption{Rotacija torusa}
\end{figure}


Preslikave oblike
\[
f(z + \Lambda) = z + \alpha + \Lambda
\]
so res avtomorfizmi torusa, saj so očitno dobro definirane in imajo
inverz
\[
f^{-1}(z + \Lambda) = z-\alpha + \Lambda.
\]
oba sta seveda holomorfna, saj je kompleksna struktura na torusu
podedovana iz kompleksne ravnine. Opazimo še, da je tudi
$f(z + \Lambda) = -z + \Lambda$ avtomorfizem torusa. To nas privede
do naslednje trditve:

\begin{trditev}
Naj bo $T \in \Aut \kvoc{\C}{\Lambda}$ poljuben avtomorfizem in
$\pi \colon \C \to \kvoc{\C}{\Lambda}$ krovna projekcija. Tedaj
obstaja taka afina funkcija $F \colon \C \to \C$, da velja
\[
T \circ \pi = \pi \circ F.
\]
\end{trditev}

\begin{proof}
Najprej opazimo, da lahko $T$ razširimo do preslikave
$f \colon \C \to \kvoc{\C}{\Lambda}$ s predpisom $f = T \circ \pi$,
kjer je $\pi$ (krovna) projekcija. Mnogoterost $\C$ je enostavno
povezana, zato po izreku o dvigu preslikave v krov
(glej~\cite[izrek~1.106]{Forstneric}) obstaja taka holomorfna
funkcija $F \colon \C \to \C$, da velja
\[
T \circ \pi = \pi \circ F.
\]
Naj bo sedaj $\alpha \in \Lambda$ poljubna točka mreže in
\[
\Phi(z) = F(z + \alpha) - F(z).
\]
Tedaj velja
\[
\pi(\Phi(z)) =
\pi(F(z + \alpha)) - \pi(F(z)) =
T(\pi(z)) - T(\pi(z + \alpha)) =
0.
\]
Sledi, da je $\Phi(\C) \subseteq \Lambda$, ker pa je to diskretna
množica in $\Phi$ zvezna funkcija, sledi, da je $\Phi$ konstantna.
Tako dobimo $F(z + \alpha) = F(z) + c$ za vsak $a \in \Lambda$,
oziroma $F'(z + \alpha) = F'(z)$. To že pomeni, da je $F'$ omejena,
zato je po Liouvilleovem izreku konstantna, kar pomeni, da je
\[
F(z) = az + b,
\]
pri čemer je $a \ne 0$.
\end{proof}

Sedaj si oglejmo, kakšne avtomorfizme take preslikave $F$
dopuščajo. Velja
\[
T(z + \Lambda) = T(\pi(z)) = \pi(F(z)) = az + b + \Lambda.
\]
Ker že vemo, da so translacije avtomorfizmi, je dovolj preveriti,
katere izmed preslikav $T(z + \Lambda) = az + \Lambda$ so
avtomorfizmi torusa.

Naj bo $\lambda \in \Lambda$ poljuben element mreže. Tedaj je
\[
\Lambda = T(\Lambda) = T(\lambda + \Lambda) = a\lambda + \Lambda,
\]
enako pa velja za inverz
$T^{-1}(z + \Lambda) = \frac{z}{a} + \Lambda$. Tako sledi, da je
$\lambda \in \Lambda$ natanko tedaj, ko je $a \lambda \in \Lambda$.
Oglejmo si $\lambda \in \Lambda \setminus \set{0}$ z najmanjšo
dolžino. Brez škode za splošnost lahko vzamemo $F(0) = 0$, kar
pomeni, da je $a \lambda \in \Lambda \setminus \set{0}$, od koder
sledi $\abs{a} \geq 1$. Z enakim premislekom za $T^{-1}$ dobimo še
$\abs{a} \leq 1$.

Če je $a = \pm 1$, dobimo avtomorfizme, ki smo jih našteli zgoraj.
Sicer sta $\lambda$ in $a \lambda$ $\R$-linearno neodvisna. Denimo,
da mreža $\Lambda$ ni generirana z $\lambda$ in $a \lambda$. To
pomeni, da obstaja element mreže $\mu \in \Lambda$, ki ga ne moremo
izraziti z $\lambda$ in $a \lambda$. To med drugim pomeni, da $\mu$
leži v notranjosti (ali na stranici) nekega romba z oglišči
$\lambda_0$, $\lambda_0 + \lambda$,
$\lambda_0 + \lambda + a \lambda$ in $\lambda_0 + a \lambda$.

\begin{figure}[!ht]
\centering
\begin{asy}
pair A = 0, B = 1, D = dir(80), C = B + D - A;
pair mu = 0.3 + 0.2I;

filldraw(A--arc(A, 1, 0, 80)--cycle, cyan+opacity(0.2));
draw(C--arc(C, 1, 180, 260)--cycle);

dot("$\lambda_0$", A, dir(225));
dot("$\lambda_0 + \lambda$", B, dir(315));
dot("$\lambda_0 + \lambda + a \lambda$", C, dir(45));
dot("$\lambda_0 + a \lambda$", D, dir(135));
dot("$\mu$", mu, dir(mu));
\end{asy}
\caption{Element mreže ne more ležati v notranjosti romba}
\label{sl:gen_mre}
\end{figure}

Sedaj ni težko videti, da velja
$\abs{\lambda_0 - \mu} < \abs{\lambda}$ ali
$\abs{\lambda_0 + \lambda + a \lambda - \mu} < \abs{\lambda}$, saj
leži v notranjosti enega izmed krožnih izsekov s središčem v
$\lambda_0$ ali $\lambda_0 + \lambda + a \lambda$ in polmerom
$\abs{\lambda}$, označenima na sliki~\ref{slika:gen_mre}. To ni
mogoče, saj sta tako $\lambda_0 - \mu$ kot
$\lambda_0 + \lambda + a \lambda - \mu$ elementa mreže, $\lambda$
pa je po dolžini najmanjši.

Ker je $a \lambda \in \Lambda$, velja tudi
$a^2 \lambda \in \Lambda$. Tako lahko izrazimo
\[
a^2 \lambda = m a \lambda + n \lambda,
\]
oziroma
\[
a^2 = ma + n.
\]
To je kvadratna enačba z realnimi koeficienti, zato sta njeni
rešitvi $a$ in $\oline{a}$. Po Vietovih formulah tako dobimo $n=1$
in $\abs{m} \leq 2$. Z obravnavo primerov dobimo
\[
a \in \set{\pm 1, \pm i,
e^{\pm \frac{i \pi}{3}}, e^{\pm \frac{2i \pi}{3}}}.
\]

\subsection{Ploskve večjih rodov}

\begin{trditev}
Naj bo $T \in \Aut M$ netrivialen avtomorfizem. Tedaj ima $T$
največ $2g + 2$ fiksnih točk.
\end{trditev}

\begin{proof}
Naj bo $P \in M$ točka, za katero je $T(P) \ne P$. Tedaj obstaja
meromorfna funkcija $f \in \mathscr{K}(M)$ z divizorjem polov
$P^r$ za nek $1 \leq r \leq g + 1$. Oglejmo si funkcijo
$h = f - f \circ T$. Njen divizor polov je očitno
$P^r (T^{-1}P)^r$. Velja torej
\[
\deg h^{-1}(0) = \deg h^{-1}(\infty) = 2r \leq 2g + 2,
\]
zato ima $h$ kvečjemu $2g + 2$ ničel. Ni težko videti, da so njene
ničle natanko fiksne točke avtomorfizma $T$.
\end{proof}

\begin{lema}
Naj bo $M$ kompaktna Riemannova ploskev roda $g \geq 2$, $W$ pa
množica njenih Weierstrassovih točk. Tedaj ta vsak avtomorfizem
$T \in \Aut M$ velja $T(W) = W$.
\end{lema}

\begin{proof}
Avtomorfizmi ohranjajo luknje.
\end{proof}

\begin{izrek}[Schwarz]
Grupe avtomorfizmov kompaktnih ploskev roda $g \geq 2$ so končne.
\end{izrek}

\begin{proof}
Po zgornji lemi sledi, da obstaja homomorfizem
$\lambda \colon \Aut M \to S_W$, kjer je $S_W$ simetrična grupa.
Dovolj je pokazati, da ima $\lambda$ končno jedro. Ločimo dva
primera.

\begin{enumerate}[a)]
\item Če $M$ ni hipereliptična, ima več kot $2g + 2$
Weierstrassovih točk. Vsak avtomorfizem, ki fiksira Weierstrassove
točke, je zato kar identiteta, zato je $\ker \lambda$ trivialno.

\item Če je $M$ hipereliptična, velja kar
$\ker \lambda = \skl{J}$, kjer je $J$ hipereliptična involucija.
Ostali avtomorfizmi imajo namreč kvečjemu $4$ fiskne točke. Ker
velja $\abs{\skl{J}} = 2$, je grupa $\Aut M$ res končna. \qedhere
\end{enumerate}
\end{proof}

\begin{izrek}[Hurwitz]
Naj bo $M$ kompaktna Riemannova ploskev roda $g \geq 2$. Tedaj je
\[
\abs{\Aut M} \leq 84(g-1).
\]
\end{izrek}

\begin{proof}
Ker je $G = \Aut M$ končna grupa, lahko tvorimo kvocient
$N = \kvoc{M}{G}$. Naj bo $\pi \colon M \to N$ kvocientna
projekcija. Ni težko videti, da je $\deg \pi = \abs{G}$, saj ima
vsaka točka $P$, ki ni fiksna točka nobenega netrivialnega
avtomorfizma, orbito velikosti $\abs{G}$. Za preslikavo $\pi$ so
razvejiščna števila točk enaka $b(P) = \abs{G_P} - 1$, pri čemer
označimo $G_P = \setb{g \in G}{g(P) = P}$. Vzemimo maksimalno
množico $\setb{P_j}{1 \leq j \leq r}$ točk, ki so fiksne točke
nekega netrivialnega avtomorfizma in imajo disjunktne orbite, in
označimo $v_j = \abs{G_{P_j}}$. Jasno je, da je velikost orbite
točke $P_j$ enaka $\frac{\abs{G}}{v_j}$, zato za razvejiščno
število preslikave velja
\[
B = \sum_{j=1}^r \frac{\abs{G}}{v_j} \cdot \br{v_j - 1},
\]
zato po izreku~\ref{iz:rie-hur} sledi
\[
2g - 2 = \abs{G} \cdot (2 \gamma - 2) +
\abs{G} \cdot \sum_{j=1}^r \br{1 - \frac{1}{v_j}},
\]
kjer je $\gamma$ rod ploskve $N$. Preostanek dokaza ločimo na tri
primere:

\begin{enumerate}[i)]
\item Velja $\gamma \geq 2$. V tem primeru mora veljati
\[
2g - 2 \geq \abs{G} \cdot 2,
\]
od koder dobimo oceno $\abs{G} \leq g-1$.

\item Velja $\gamma = 1$. Če je $r=0$, dobimo $g=1$, v nasprotnem
primeru pa velja
\[
\sum_{j=1}^r \br{1 - \frac{1}{v_j}} \geq \frac{1}{2},
\]
od koder sledi $\abs{G} \leq 4(g-1)$.

\item Velja $\gamma = 0$. Od tod lahko enačbo prepišemo v
\begin{equation}\label{eq:hur}
2g - 2 =
\abs{G} \cdot \br{\sum_{j=1}^r \br{1 - \frac{1}{v_j}} - 2}.
\end{equation}
Veljati mora $r \geq 3$, saj je v nasprotnem primeru desna stran
enačbe negativna. Če je $r \geq 5$, sledi
\[
\sum_{j=1}^r \br{1 - \frac{1}{v_j}} \geq \frac{5}{2},
\]
zato je $\abs{G} \leq 4(g-1)$. Enostavno se lahko znebimo tudi
primera $r=4$. V tem primeru namreč ne morejo vsi $v_j$ biti enaki
$2$, saj bi tedaj desna stran enačbe~\eqref{eq:hur} bila negativna.
Tako lahko ocenimo
\[
\sum_{j=1}^4 \br{1 - \frac{1}{v_j}} \geq
3 \cdot \frac{1}{2} + \frac{2}{3} =
\frac{13}{6},
\]
zato velja ocena $\abs{G} \leq 12(g-1)$.

Preostane še primer $r=3$. Naj bo $2 \leq v_1 \leq v_2 \leq v_3$.
Najprej opazimo, da je $v_2 \geq 3$, saj bi v nasprotnem primeru
desna stran enačbe~\eqref{eq:hur} bila negativna. Iz enakega
razloga dobimo tudi $v_3 > 3$. Če je $v_3 \geq 7$, tako dobimo
\[
\sum_{j=1}^3 \br{1 - \frac{1}{v_j}} \geq
\frac{1}{2} + \frac{2}{3} + \frac{6}{7} = \frac{85}{42},
\]
od koder dobimo oceno iz trditve izreka, to je
$\abs{G} \leq 84(g-1)$.

Preostane nam še možnost $v_3 \leq 6$. Pri vsakem izmed teh končno
mnogo primerov dobimo, da je desna stran enačbe~\eqref{eq:hur}
negativna ali pa sledi ocena $\abs{G} \leq 40(g-1)$. \qedhere
\end{enumerate}
\end{proof}

Za konec omenimo še, da je enakost $\abs{G} = 84 (g-1)$ dejansko
dosežena za nekatere vrednosti $g$, na primer $g = 3$ in $g = 7$.
