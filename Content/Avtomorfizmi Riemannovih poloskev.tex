\section{Avtomorfizmi Riemannovih poloskev}

\subsection{Sfere in torusi}

Za določanje grupe avtomorfizmov Riemannovih ploskev so pomembne
njihove topološke lastnosti -- vsak avtomorfizem je namreč tudi
homeomorfizem. Iz geometrijske topologije vemo, da je vsaka
orientabilna kompaktna ploskev homeomorfna vsoti $g$ torusov.
Številu $g$ pravimo rod ploskve.

Najprej si oglejmo ploskve z ničelnim rodom -- topološko so to kar
sfere. V prejšnjih razdelkih smo ugotovili, da je grupa
avtomorfizmov Riemannove sfere enaka
\[
\Aut \br{\rs} = \setb{\frac{az + b}{cz + d}}{ad - bc = 1}.
\]

Vemo pa, da je grupa avtomorfizmov odvisna ne samo od topoloških
lastnosti objekta, ampak tudi njegove kompleksne strukture.
%TODO To ni problem

Naslednji izziv so ploskve z rodom $g=1$ -- torusi. Za toruse
IZREK ne velja, zato imamo več različnih grup avtomorfizmov.
Oglejmo si, kako jih dobimo:
%TODO Kvocienti in stvari

\subsection{Ploskve večjih rodov}

\begin{trditev}
Naj bo $T \in \Aut M$ netrivialen avtomorfizem. Tedaj ima $T$
največ $2g + 2$ fiksnih točk.
\end{trditev}

\begin{proof}
Naj bo $P \in M$ točka, za katero je $T(P) \ne P$. Tedaj obstaja
meromorfna funkcija $f \in \mathscr{K}(M)$ s polarnim deliteljem
$P^r$ za nek $1 \leq r \leq g + 1$. Oglejmo si funkcijo
$h = f - f \circ T$. Njen polarni delitelj je očitno
$P^r (T^{-1}P)^r$. Velja torej
\[
\deg h^{-1}(0) = \deg h^{-1}(\infty) = 2r \leq 2g + 2,
\]
zato ima $g$ kvečjemu $2g + 2$ ničel. Ni težko videti, da so njene
ničle natanko fiksne točke avtomorfizma $T$.
\end{proof}

\begin{lema}
Naj bo $M$ kompaktna Riemannova ploskev roda $g \geq 2$, $W$ pa
množica njenih Weierstrassovih točk. Tedaj ta vsak avtomorfizem
$T \in \Aut M$ velja $T(W) = W$.
\end{lema}

\begin{proof}
Avtomorfizmi ohranjajo GAPE.
\end{proof}

\begin{izrek}[Schwarz]
Grupe avtomorfizmov kompaktnih ploskev roda $g \geq 2$ so končne.
\end{izrek}

\begin{proof}
Po zgornji lemi sledi, da obstaja homomorfizem
$\lambda \colon \Aut M \to S_W$, kjer je $S_W$ simetrična grupa.
Dovolj je pokazati, da ima $\lambda$ končno jedro. Ločimo dva
primera.

\begin{enumerate}[a)]
\item Če $M$ ni hipereliptična, ima več kot $2g + 2$
Weierstrassovih točk. Vsak avtomorfizem, ki fiksira Weierstrassove
točke, je zato kar identiteta, zato je $\ker \lambda$ trivialno.

\item Če je $M$ hipereliptična, velja kar
$\ker \lambda = \skl{J}$, kjer je $J$ hipereliptična involucija,
velja pa $\abs{\skl{J}} = 2$. \qedhere
\end{enumerate}
\end{proof}

%TODO Hurwitz
