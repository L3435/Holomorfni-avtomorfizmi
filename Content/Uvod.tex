\section{Uvod}

V diplomskem delu si bomo ogledali simetrije določenih
geometrijskih objektov. Specifično nas bodo zanimale konformne
preslikave, torej tiste, ki ohranjajo kote. Ekvivalentno so to
holomorfne preslikave z neničelnim odvodom.

V prvem delu se bomo osredotočili na odprte množice v kompleksni
ravnini. Glavna ideja bo, da vsak avtomorfizem dovolj regularnega
območja razširimo do avtomorfizma Riemannove sfere, nato pa bomo
glede na njegove geometrijske lastnosti omejili število
avtomorfizmov. Glavni rezultat tega dela bo Heinsov izrek iz leta
1946, ki pravi, da za vsa dovolj velika naravna števila $n$ območja
z $n$ luknjami premorejo največ $2n$ holomorfnih avtomorfizmov.

V naslednjem razdelku bomo uvedli pojem Riemannovih ploskev.
Raziskali bomo posledice Riemann-Rochovega izreka, ki ga je dokazal
G.~Roch leta 1865. Z njim bomo poiskali točke na kompaktnih
Riemannovih ploskvah, v katerih imajo meromorfne funkcije lahko pole
majhne stopnje, in jih analizirali.

Na koncu si bomo ogledali še holomorfne avtomorfizme kompaktnih
Riemannovih ploskev. Dokazali bomo Schwarzov in Hurwitzev izrek iz
let 1878 in 1893, ki omejujeta število holomorfnih avtomorfizmov
Riemannovih ploskev rodov $g \geq 2$ na $84 (g-1)$.
