\section{Holomorfni avtomorfizmi v kompleksni ravnini}

\subsection{Enostavno povezana območja}

Da lahko govorimo o holomorfnih funkcijah, se bomo omejili na
odprte podmnožice kompleksne ravnine.

\begin{definicija}
\emph{Holomorfen avtomorfizem} odprte množice $\Omega \subseteq \C$
je bijektivna holomorfna preslikava $f \colon \Omega \to \Omega$ s
holomorfnim inverzom.
\end{definicija}

Opazimo, da je zadosten pogoj že to, da je $f$ bijektivna in
holomorfna. Opazimo še, da množica avtomorfizmov nekega območja
tvori grupo z operacijo kompozitum. To grupo označimo z
$\Aut(\Omega)$.

Prepričamo se lahko, da lahko avtomorfizme nepovezanih množic
opišemo z avtomorfizmi, ki komponente med seboj permutirajo. V
nadaljevanju se bomo tako omejili na povezane množice.

\begin{definicija}
\emph{Območje} v kompleksni ravnini $\C$ je vsaka odprta povezana
množica.
\end{definicija}

\begin{primer}
Kompleksna ravnina je območje v $\C$. Njena grupa avtomorfizmov
je enaka
\[
\Aut(\C) = \setb{z \mapsto az + b}{a \ne 0}. \qedhere
\]
\end{primer}

\begin{primer}
Naj bo $\dsk$ odprt enotski disk v $\C$. Tedaj je
\[
\Aut(\dsk) =
\setb{z \mapsto e^{i \theta} \cdot \frac{z-a}{1 - \oline{a} z}}
{a \in \dsk \land \theta \in [0, 2\pi)}. \qedhere
\]
\end{primer}

\begin{figure}[!ht]
\centering

\begin{asy}
int ITER = 1000;
real STEP = 0.3;
pair a = (0.6,-0.3);
real theta = 0.3pi;
pair aut(pair X){
  return X;
}

path Re(real y, pair A(pair)){
  guide g;
  for(int i = 0; i < 1000; ++i){
    g = g--(A((2*sqrt(1-y*y)*i/ITER - sqrt(1-y*y),y)));
  }
  path p = g;
  return p;
}

path Im(real x, pair A(pair)){
  guide g;
  for(int i = 0; i < 1000; ++i){
    g = g--(A((x,2*sqrt(1-x*x)*i/ITER - sqrt(1-x*x))));
  }
  path p = g;
  return p;
}
draw(circle((0,0),1));
for(real t = -0.9; t <= 1; t += STEP){
	draw(Re(t, aut),red);
	draw(Im(t, aut),blue);
}
dot(a);
xaxis("$\Re$",Ticks(Label(fontsize(8pt)),new real[]{}),xmin=-1.25,xmax=1.25);
yaxis("$\Im$",Ticks(Label(fontsize(8pt)),new real[]{}),ymin=-1.25,ymax=1.25);
\end{asy}
\hfill
\begin{asy}
int ITER = 1000;
real STEP = 0.3;
pair a = (0.6,-0.3);
real theta = 0.3pi;
pair aut(pair X){
  return exp(I*theta) * (X-a) / (1-conj(a)*X);
}

path Re(real y, pair A(pair)){
  guide g;
  for(int i = 0; i < 1000; ++i){
    g = g--(A((2*sqrt(1-y*y)*i/ITER - sqrt(1-y*y),y)));
  }
  path p = g;
  return p;
}

path Im(real x, pair A(pair)){
  guide g;
  for(int i = 0; i < 1000; ++i){
    g = g--(A((x,2*sqrt(1-x*x)*i/ITER - sqrt(1-x*x))));
  }
  path p = g;
  return p;
}
draw(circle((0,0),1));
for(real t = -0.9; t <= 1; t += STEP){
	draw(Re(t, aut),red);
	draw(Im(t, aut),blue);
}
dot(aut((0,0)));
xaxis("$\Re$",Ticks(Label(fontsize(8pt)),new real[]{}),xmin=-1.25,xmax=1.25);
yaxis("$\Im$",Ticks(Label(fontsize(8pt)),new real[]{}),ymin=-1.25,ymax=1.25);
\end{asy}
\caption{Primer avtomorfizma diska. Označeni sta točki $f^{-1}(0)$
in $f(0)$.}
\end{figure}

Izkaže se, da smo s tem do izomorfizma natančno opisali grupe
avtomorfizmov vseh povezanih in enostavno povezanih množic v $\C$.
Velja namreč naslednja lema:

\begin{lema}
Naj bosta $\Omega_1$ in $\Omega_2$ biholomorfno ekvivalentni
območji v $\C$. Tedaj je $\Aut(\Omega_1) \cong \Aut(\Omega_2)$.
\end{lema}

\begin{proof}
Naj bo $f \colon \Omega_1 \to \Omega_2$ biholomorfna preslikava.
Sedaj definiramo preslikavo
$\Phi \colon \Aut(\Omega_1) \to \Aut(\Omega_2)$ s predpisom
\[
\Phi(\phi) = f^{-1} \circ \phi \circ f.
\]
Ker je s predpisom
\[
\Phi^{-1}(\psi) = f \circ \psi \circ f^{-1}
\]
očitno podan predpis inverza preslikave $\Phi$, je ta bijektivna.
Velja pa
\[
\Phi(\phi \circ \psi) = f^{-1} \circ \phi \circ \psi \circ f =
\br{f^{-1} \circ \phi \circ f} \circ
\br{f^{-1} \circ \psi \circ f} =
\Phi(\phi) \circ \Phi(\psi),
\]
zato je $\Phi$ homomorfizem.
\end{proof}

Spomnimo se na Riemannov upodobitveni izrek, ki pravi, da je
vsako povezano in enostavno povezano območje v kompleksni ravnini
ali biholomorfno ekvivalentno $\dsk$ ali pa kar enako $\C$. Grupe
avtomorfizmov povezanih in enostavno povezanih območij so do
izomorfizma natančno tako le $\Aut(\dsk)$ in $\Aut(\C)$.

Omenimo še, da lahko kompleksno ravnino dopolnimo do Riemannove
sfere $\rs$. Njeni avtomorfizmi so Möbiusove transformacije, torej
\[
\Aut \br{\rs} =
\setb{z \mapsto \frac{az + b}{cz + d}}{ad - bc = 1}.
\]

\subsection{Punktirani diski in kolobarji}

Po obravnavi enostavno povezanih območij so naslednji korak
območja z ">luknjami"<. Eden izmed osnovnejših primerov takih
območij je punktiran disk
$\dsk_\alpha = \dsk \setminus \set{\alpha}$.

Disk $\dskx = \dsk \setminus \set{0}$ je biholomorfno
ekvivalenten vsakemu punktiranemu disku, saj je preslikava
$f \colon \dsk_\alpha \to \dskx$ s predpisom
\[
f(z) = \frac{z - \alpha}{1 - \oline{\alpha} z}
\]
biholomorfna. Sledi, da je
$\Aut(\dsk_\alpha) \cong \Aut(\dskx)$.

\begin{trditev}
Za punktiran disk velja
\[
\Aut(\dskx) =
\setb{z \mapsto e^{i \pi \theta} z}{\theta \in [0, 2\pi)}.
\]
\end{trditev}

\begin{proof}
Naj bo $f \colon \dskx \to \dskx$ poljuben avtomorfizem. Tedaj je
točka $0$ izolirana singularnost funkcije $f$. Ker je $f$ omejena,
je to odpravljiva singularnost.

Naj bo $\widetilde{f} \colon \dsk \to \C$ funkcija, ki jo dobimo
tako, da $f$ razširimo na celoten disk. Predpostavimo, da velja
$\widetilde{f}(0) \ne 0$. Ker je $\widetilde{f}$ holomorfna, je
odprta, zato sledi $\abs{\widetilde{f}(0)} < 1$. Oglejmo si točko
$\alpha \ne 0$, za katero je $f(\alpha) = \widetilde{f}(0)$.
Izberemo si lahko disjunktni okolici $U$ in $V$ točk $0$ in
$\alpha$. Ker je $\widetilde{f}$ odprta, je odprta tudi množica
$W = \widetilde{f}(U) \cap \widetilde{f}(V)$. Hitro opazimo, da
velja $\widetilde{f}(0) \in W$, zato je ta množica neprazna.
Sledi, da je $W$ neskončna, kar je protislovje, saj velja
$f(U \setminus \set{0}) \cap f(V) = \emptyset$.

Sledi, da je $\widetilde{f}(0) = 0$, zato je $\widetilde{f}$
avtomorfizem diska. Tako dobimo inkluzijo
\[
\Aut(\dskx) \subseteq \setb{f \in \Aut(\dsk)}{f(0) = 0}.
\]
Ni težko preveriti, da velja tudi obratna inkluzija, oziroma
\[
\Aut(\dskx) =
\setb{z \mapsto e^{i \pi \theta} z}{\theta \in [0, 2\pi)}.
\qedhere
\]
\end{proof}

Kaj pa se zgodi, če število lukenj povečamo? Naj bo
$\dsk_2 = \dsk \setminus \set{0, \alpha}$.\footnote{Podobno kot
prej lahko privzamemo, da je ena izmed lukenj enaka $0$.} Po
enakem razmisleku kot prej ugotovimo, da za vsak avtomorfizem
$f \in \Aut(\dsk_2)$ velja $f(0) \in \dsk$ in hkrati
$f(0) \not \in \dsk_2$. Enako velja za točko $\alpha$. Sedaj ni
težko videti, da velja
\[
\Aut(\dsk_2) =
\skl{z \mapsto \frac{\alpha - z}{1 - \oline{\alpha} z}} \cong
\Z_2,
\]
saj je avtomorfizem diska natančno določen z dvema točkama.

Sedaj si oglejmo še kolobar $R = \dsk \setminus \dsk(r)$. Naj bo
$f \colon R \to R$ avtomorfizem. Izkaže se, da se $f$ zvezno
razširi na $\partial R$. Ker lahko $f$ komponiramo s preslikavo
$\varphi \colon z \mapsto \frac{\rho}{z}$, lahko predpostavimo, da
je $f(\partial \dsk) = \partial \dsk$.

Naj bo
\[
u(z) = \log \abs{f(z)} - \log \abs{z}.
\]
Ker je logaritem harmonična funkcija, je $\lapl u = 0$. Ker je
$\eval{u}{\partial R}{} = 0$, je po principu maksima $u = 0$. Tako
sledi $\abs{f(z)} = \abs{z}$ in
\[
\abs{\frac{f(z)}{z}} = 1.
\]
Ker je $\frac{f(z)}{z}$ holomorfna in so nekonstantne holomorfne
preslikave odprte, sledi, da je $\frac{f(z)}{z}$ konstantna. Tako
so vsi avtomorfizmi kolobarja kar $z \mapsto e^{i \theta} z$ in
$z \mapsto e^{i \theta} \frac{r}{z}$.

\begin{figure}[!ht]
\centering

\begin{asy}
int ITER = 1000;
real STEP = 0.3;
pair a = (0.6,-0.3);
real theta = 0.3pi;
real r = 0.4;

pair aut(pair X){
  return X;
}

path Re(real y, pair A(pair), bool first = true){
  guide g;
  if(y * y >= r * r) {
    if(!first) return (0,0);
    real d = 2 * sqrt(1-y*y);
    real start = -sqrt(1-y*y);
    for(int i = 0; i < 1000; ++i){
      g = g--(A((d*i/ITER + start, y)));
    }	
  }
  else {
    real d = sqrt(1-y*y) - sqrt(r*r-y*y);
    real start = sqrt(r*r-y*y);
    if(!first) start = -sqrt(1-y*y);
    for(int i = 0; i < 1000; ++i){
      g = g--(A((d*i/ITER + start, y)));
    }
  }
  path p = g;
  return p;
}

path Im(real x, pair A(pair), bool first = true){
  guide g;
  if(x * x >= r * r) {
    if(!first) return (0,0);
    real d = 2 * sqrt(1-x*x);
    real start = -sqrt(1-x*x);
    for(int i = 0; i < 1000; ++i){
      g = g--(A((x,d*i/ITER + start)));
    }	
  }
  else {
    real d = sqrt(1-x*x) - sqrt(r*r-x*x);
    real start = sqrt(r*r-x*x);
    if(!first) start = -sqrt(1-x*x);
    for(int i = 0; i < 1000; ++i){
      g = g--(A((x,d*i/ITER + start)));
    }
  }
  path p = g;
  return p;
}
draw(circle((0,0),1));
draw(circle(0,r));
for(real t = -0.9; t <= 1; t += STEP){
	draw(Re(t, aut),red);
	draw(Re(t, aut, false),red);
	draw(Im(t, aut),blue);
	draw(Im(t, aut, false),blue);
}
xaxis("$\Re$",Ticks(Label(fontsize(8pt)),new real[]{}),xmin=-1.25,xmax=1.25);
yaxis("$\Im$",Ticks(Label(fontsize(8pt)),new real[]{}),ymin=-1.25,ymax=1.25);
\end{asy}
\hfill
\begin{asy}
int ITER = 1000;
real STEP = 0.3;
pair a = (0.6,-0.3);
real theta = 0.3pi;
real r = 0.4;

pair aut(pair X){
  return exp(I*theta) * r / X;
}

path Re(real y, pair A(pair), bool first = true){
  guide g;
  if(y * y >= r * r) {
    if(!first) return (0,0);
    real d = 2 * sqrt(1-y*y);
    real start = -sqrt(1-y*y);
    for(int i = 0; i < 1000; ++i){
      g = g--(A((d*i/ITER + start, y)));
    }	
  }
  else {
    real d = sqrt(1-y*y) - sqrt(r*r-y*y);
    real start = sqrt(r*r-y*y);
    if(!first) start = -sqrt(1-y*y);
    for(int i = 0; i < 1000; ++i){
      g = g--(A((d*i/ITER + start, y)));
    }
  }
  path p = g;
  return p;
}

path Im(real x, pair A(pair), bool first = true){
  guide g;
  if(x * x >= r * r) {
    if(!first) return (0,0);
    real d = 2 * sqrt(1-x*x);
    real start = -sqrt(1-x*x);
    for(int i = 0; i < 1000; ++i){
      g = g--(A((x,d*i/ITER + start)));
    }	
  }
  else {
    real d = sqrt(1-x*x) - sqrt(r*r-x*x);
    real start = sqrt(r*r-x*x);
    if(!first) start = -sqrt(1-x*x);
    for(int i = 0; i < 1000; ++i){
      g = g--(A((x,d*i/ITER + start)));
    }
  }
  path p = g;
  return p;
}
draw(circle((0,0),1));
draw(circle(0,r));
for(real t = -0.9; t <= 1; t += STEP){
	draw(Re(t, aut),red);
	draw(Re(t, aut, false),red);
	draw(Im(t, aut),blue);
	draw(Im(t, aut, false),blue);
}
xaxis("$\Re$",Ticks(Label(fontsize(8pt)),new real[]{}),xmin=-1.25,xmax=1.25);
yaxis("$\Im$",Ticks(Label(fontsize(8pt)),new real[]{}),ymin=-1.25,ymax=1.25);
\end{asy}
\caption{Primer avtomorfizma kolobarja.}
\end{figure}


\subsection{Avtomorfizmi \texorpdfstring{$p$}{p}-povezanih območij}

Oglejmo si avtomorfizme območja
$\Omega = \C \setminus \setb{x_i}{1 \leq i \leq p}$, pri čemer
brez škode za splošnost vzamemo $x_p = \infty$. Za $p = 1$ dobimo
kompleksno ravnino. Pri $p = 2$ lahko brez škode za splošnost
vzamemo $x_1 = 0$. Ni težko videti, da so vsi avtomorfizmi oblike
$z \mapsto e^{i \theta} \cdot z^{\pm 1}$.

Sedaj si oglejmo še primer $p > 2$. Preverimo lahko, da se vsak
avtomorfizem $\Omega$ razširi do avtomorfizma Riemannove ploskve,
ki permutira točke $x_i$. Ker je vsaka Möbiusova transformacija
enolično določena s tremi točkami, je moč grupe $\Aut(\Omega)$ tako
omejena s $p (p-1) (p-2)$.

Izkaže se, da lahko to mejo še bistveno izboljšamo. Znano je
namreč, da je vsaka končna podgrupa $\Aut(\rs)$ konjugirana
podgrupi grupe rotacij $\SO_3$ \cite{Lyndol}. Vse končne podgrupe
$\SO_3$ so natanko rotacijske, diedrske, tetraedrska, oktaedrska in
ikozaedrska \cite{Artin_1991}. Preverimo lahko, da je za $p = 4$
največja možna moč grupe avtomorfizmov enaka $12$, za $p = 6$ in
$p = 8$ dobimo $24$, za $p = 12$ in $p = 20$ pa $60$. Za vse ostale
$p$ je največja grupa simetrij kar dierska, zato je
$\abs{\Aut(\Omega)} \leq 2p$.

%TODO Podrobnejša razlaga

Oglejmo si še primer, ko robne komponente niso nujno točke.

\begin{izrek}
Naj bo $\Omega$ območje na Riemannovi sferi, katerega robne
komponente sestavlja $p$ točk ali Jordanovih krivulj. Tedaj za
moč grupe $\Aut \Omega$ veljajo enake ocene kot zgoraj.
\end{izrek}

\begin{proof}
Po posplošitvi Riemannovega upodobitvenega izreka lahko
predpostavimo, da je $\Omega$ kar enotski disk, ki mu odstranimo
nekaj točk in diskov. Po Carathéodoryjevem izreku
\cite[izrek~5.1.1 in opomba~5.1.2]{krantz} sledi, da avtomorfizem
permutira robne komponente območja $\Omega$. Tako obstaja
homomorfizem $\lambda \colon \Aut \Omega \to S_{\partial \Omega}$,
kjer je $S_{\partial{\Omega}}$ permutacijska grupa. Jedro
homomorfizma so natanko tisti avtomorfizmi, ki fiksirajo vsako
robno komponento posebej. Pokažimo, da je jedro trivialno.

Naj bo $T \in \ker \lambda$ poljuben avtomorfizem. Najprej se lahko
znebimo točkastih robnih komponent, saj se te preslikajo same vase.
Z uporabo Schwarzovega zrcaljenja lahko $T$ razširimo preko vsake
druge robne komponente. Enostavno je preveriti, da se pri tem
ohrani injektivnost preslikave. V limiti nam v vsaki komponenti
$\Omega^\mathsf{c}$ ostane samo ena točka. V vseh omejenih
komponentah je to odpravljiva singularnost, v neomejeni pa ali
odpravljiva singularnost ali pol prve stopnje. V obeh primerih se
$T$ razširi do avtomorfizma Riemannove sfere. Zanimajo nas torej
avtomorfizmi Riemannove sfere, ki fiksirajo vse robne komponente
$\Omega$. V nadaljevanju na točkaste robne komponente glejmo kot na
diske z radijem $0$.

Naj bo $T = \frac{az + b}{cz + d}$. Sedaj obravnavajmo dva primera.

\begin{enumerate}[i)]
\item Velja $c=0$. V tem primeru je $T$ afina transformacija. To
implicira, da $T$ poleg robnih komponent fiksira tudi njihova
središča. Ker sta to vsaj dve različni točki, ima $T$ dve fiksni
točki, zato velja kar $T = \id$.

\item Velja $c \ne 0$. Tedaj je $T$ kompozitum inverzije v točki
$z_0 = -\frac{d}{c}$, rotacije in translacije. Ker mora inverzija
ohranjati radije robnih komponent (preostali preslikavi sta namreč
izometriji), morajo vse biti ortogonalne na neko krožnico s
središčem v $z_0$.\footnote{Ne morejo imeti središča v $z_0$, saj
bi tedaj $T$ območje $\Omega$ slikal v notranjost te krožnice.}
Sedaj opazimo, da smo prišli do protislovja, saj je inverzija
obrnila orientacijo središč robnih komponent, česar pa ne moremo
popraviti z rotacijo in translacijo.
\end{enumerate}

\begin{figure}[!ht]
\centering
\begin{asy}
path orth(pair X, string s){
	dot(s, X, dir(X));
	return shift(X) * scale(sqrt(abs(X)^2-1)) * unitcircle;
}

draw(circle(0,1), red+dashed);

path[] g;
g = g ^^ orth(1.8, "$0$");
g = g ^^ reverse(orth(1.1*dir(25), "$S_A$"));
dot("$S_B$", dir(-5), 2dir(-5));
g = g ^^ reverse(orth(1.08*dir(-32), "$S_C$"));
filldraw(g, cyan+opacity(0.2));
label("$\Omega$", shift(1.8) * scale(sqrt(1.8^2-1)) * dir(45), align = N+E);

dot("$z_0$", 0, dir(180));
\end{asy}
\qquad
\begin{asy}
path orth(pair X, string s){
	dot(s, X, dir(X));
	return shift(X) * scale(sqrt(abs(X)^2-1)) * unitcircle;
}

draw(circle(0,1), red+dashed);

path[] g;
g = g ^^ orth(1.8, "$0$");
g = g ^^ reverse(orth(1.1*dir(-25), "$S_A'$"));
dot("$S_B'$", dir(5), 2dir(5));
g = g ^^ reverse(orth(1.08*dir(32), "$S_C'$"));
filldraw(g, cyan+opacity(0.2));
label("$\Omega$", shift(1.8) * scale(sqrt(1.8^2-1)) * dir(45), align = N+E);

dot("$z_0$", 0, dir(180));
\end{asy}
\caption{Inverzija prezrcali robne komponente}
\end{figure}


Sedaj si izberimo poljubno točko $z \not \in \Omega$ in si oglejmo
njeno orbito, torej množico $A = \setb{T(z)}{T \in \Aut \Omega}$.
Po enakem razmisleku kot zgoraj se namreč vsak avtomorfizem
$\Omega$ razširi do avtomorfizma kompleksne ravnine. V vsaki
komponenti $\Omega^\mathsf{c}$ imamo kvečjemu en element množice
$A$. Sedaj si izberimo poljubno točko v eni izmed komponent, ki je
množica $A$ ne obišče, in postopek ponovimo. Tako na koncu dobimo
množico $S = \setb{z_i}{1 \leq i \leq p}$, pri čemer je v vsaki
komponenti $\Omega^\mathsf{c}$ natanko ena točka. Poljuben
avtomorfizem $T \in \Aut \Omega$ se torej razširi do avtomorfizma
območja $\Omega \setminus S$, s tem pa je izrek dokazan.
\end{proof}

%TODO Vir izreka
