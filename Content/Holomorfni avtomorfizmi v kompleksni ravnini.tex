\section{Holomorfni avtomorfizmi v kompleksni ravnini}

\subsection{Enostavno povezana območja}

\begin{definicija}
\emph{Območje} v kompleksni ravnini $\C$ je vsaka odprta povezana
množica.
\end{definicija}

\begin{definicija}
\emph{Holomorfni avtomorfizem} območja $\Omega$ je bijektivna
holomorfna preslikava $f \colon \Omega \to \Omega$ s holomorfnim
inverzom.
\end{definicija}

Opazimo, da je zadosten pogoj že to, da je $f$ bijektivna z
neničelnim odvodom. Opazimo še, da množica avtomorfizmov nekega
območja tvori grupo z operacijo kompozitum. To grupo označimo z
$\Aut(\Omega)$.

\begin{primer}
Kompleksna ravnina je območje v $\C$. Njena grupa avtomorfizmov
je enaka
\[
\Aut(\C) = \setb{az + b}{a \ne 0}. \qedhere
\]
\end{primer}

\begin{primer}
Naj bo $\dsk$ odprt enotski disk v $\C$. Tedaj je
\[
\Aut(\dsk) =
\setb{e^{i \theta} \cdot \frac{z-a}{1 - \oline{a} z}}
{a \in \dsk \land \theta \in [0, 2\pi)}. \qedhere
\]
\end{primer}

Izkaže se, da smo s tem do izomorfizma natančno opisali grupe
avtomorfizmov vseh povezanih in enostavno povezanih množic v $\C$.
Velja namreč naslednja lema:

\begin{lema}
Naj bosta $\Omega_1$ in $\Omega_2$ biholomorfno ekvivalentni
območji v $\C$. Tedaj je $\Aut(\Omega_1) \cong \Aut(\Omega_2)$.
\end{lema}

\begin{proof}
Naj bo $f \colon \Omega_1 \to \Omega_2$ biholomorfna preslikava.
Sedaj definiramo preslikavo
$\Phi \colon \Aut(\Omega_1) \to \Aut(\Omega_2)$ s predpisom
\[
\Phi(\phi) = f^{-1} \circ \phi \circ f.
\]
Ker je s predpisom
\[
\Phi^{-1}(\psi) = f \circ \psi \circ f^{-1}
\]
očitno podan predpis inverza preslikave $\Phi$, je ta bijektivna.
Velja pa
\[
\Phi(\phi \circ \psi) = f^{-1} \circ \phi \circ \psi \circ f =
\br{f^{-1} \circ \phi \circ f} \circ
\br{f^{-1} \circ \psi \circ f} =
\Phi(\phi) \circ \Phi(\psi),
\]
zato je $\Phi$ homomorfizem.
\end{proof}

Spomnimo se na Riemannov upodobitveni izrek, ki pravi, da je
vsako povezano in enostavno povezano območje v kompleksni ravnini
ali biholomorfno ekvivalentno $\dsk$ ali pa kar enako $\C$. Grupe
avtomorfizmov povezanih in enostavno povezanih območij so do
izomorfizma natančno tako le $\Aut(\dsk)$ in $\Aut(\C)$.

Omenimo še, da lahko kompleksno ravnino dopolnimo do Riemannove
sfere $\rs$. Njeni avtomorfizmi so Möbiusove transformacije, torej
\[
\Aut \br{\rs} = \setb{\frac{az + b}{cz + d}}{ad - bc = 1}.
\]

\subsection{Kolobarji in punktirani diski}

Po obravnavi enostavno povezanih območij so naslednji korak
območja z ">luknjami"<. Najosnovnejši tak primer je seveda
kolobar.

%TODO

Opazimo, da se pri velikem številu lukenj grupa avtomorfizmov
bistveno zmanjša -- enostavno povezana območja imajo neskončno
avtomorfizmov, prav tako območja z eno luknjo.
