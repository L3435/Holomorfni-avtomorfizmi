\section{Riemannove ploskve}

\subsection{Gladke in kompleksne mnogoterosti}

%TODO Definicije gladkih mnogoterosti

\begin{figure}[!ht]
\centering
\begin{asy}
size(10cm);
guide add(guide g, pair[] list) {
	for(pair P : list) g = g..P;
	return g;
}

void center(pair O) {
	guide lower = O+(-0.8,-0.05)..O+(0,-0.15)..O+(0.8,-0.05);
	guide upper = O+(-0.8,-0.05)..O+(0,0.2)..O+(0.8,-0.05);
	draw(upper);
	draw(O+(-1.1,0.05)..lower..O+(1.1,0.05));
}

pair[] zgoraj = {(-2,0.65),(-1,0.86),(0,1),(1,0.86),(2,0.65)};
pair[] desno = {(3,0.86),(4,1),(6,0),(4,-1),(3,-0.86)};
pair[] spodaj = {(2,-0.65),(1,-0.86),(0,-1),(-1,-0.86),(-2,-0.65)};
pair[] levo = {(-3,-0.86),(-4,-1),(-6,0),(-4,1),(-3,0.86)};
guide g;
g = add(g, zgoraj);
g = add(g, desno);
g = add(g, spodaj);
g = add(g, levo);
draw(g..cycle);
center(0);
center(4);
center(-4);
\end{asy}
\caption{Ploskev roda $g=3$}
\end{figure}


\begin{definicija}
\emph{Meromorfen $q$-diferencial} $\omega$ Riemannove ploskve je
dodelitev meromorfne funkcije $f$ vsaki lokalni koordinati, pri
čemer je $f(z)\,dz^q$ neodvisna od lokalne koordinate. Meromorfnim
$1$-diferencialom pravimo kar \emph{meromorfni diferenciali}.
\end{definicija}

Naj bosta $(U, \varphi)$ in $(V, \psi)$ lokalni karti, za kateri
velja $U \cap V \ne \emptyset$. Če jima meromorfen $q$-diferencial
$\omega$ priredi funkciji $f_U$ in $f_V$, mora tako veljati
\[
f_U = f_V \cdot \br{\br{\psi \circ \varphi^{-1}}'}^q.
\]
Opazimo, da je $q$-ta potenca meromorfnega diferenciala meromorfen
$q$-diferencial.

\begin{trditev}
Naj bosta $\alpha$ in $\beta$ meromorfna $q$-diferenciala. Tedaj je
$\frac{\alpha}{\beta}$ meromorfna funkcija.
\end{trditev}

\begin{proof}
Z zgornjimi oznakami velja
\[
\frac{\alpha_U}{\beta_U} =
\frac{\alpha_V \cdot \br{\br{\psi \circ \varphi^{-1}}'}^q}
{\beta_V \cdot \br{\br{\psi \circ \varphi^{-1}}'}^q} =
\frac{\alpha_V}{\beta_V}.
\]
Kvocient $\frac{\alpha}{\beta}$ tako ni odvisen od lokalnih
koordinat.
\end{proof}

Očitno velja tudi obratno -- če je $\alpha$ meromorfen
$q$-diferencial in $f$ meromorfna funkcija, je tudi $f \alpha$
meromorfen $q$-diferencial.

\begin{trditev}
\label{td:deg}
Naj bo $f \colon M \to N$ nekonstantna holomorfna preslikava med
kompaktnima Riemannovima ploskvama. Tedaj obstaja naravno število
$m$, za katero $f$ doseže vsako točko $Q \in N$ natanko
$m$-krat.\footnote{Šteto z večkratnostmi.}
\end{trditev}

\begin{proof}
Iz kompleksne analize vemo, da za vsako točko $P \in M$ obstajajo
take lokalne koordinate $\tilde{z}$, da je
$f(\tilde{z}) = f(P) + \tilde{z}^n$. Število $n-1$ označimo z
$b(P)$ in mu pravimo BRANCHING NUMBER. To je seveda dobro
definirano.

Za vsako naravno število $m$ naj bo
\[
\Sigma_m =
\setb{X \in N}{\sum_{f(P)=X} (b(P)+1) \geq m}.
\]
Označimo še
\[
\varphi(X) = \sum_{f(P) = X} (b(P)+1).
\]
Vse množice $\Sigma_m$ so odprte -- če je $b(P) = n-1$, lahko v
lokalnih koordinatah zapišemo $f(\tilde{z}) = \tilde{z}^n$. Enačba
$f(\tilde{z}) = \varepsilon$ ima tako natanko $n$ rešitev, zato za
okolico $U$ točke $P$ velja
\[
b(P) + 1 = \sum_{Q \in U \cap f^{-1}(P')} (b(Q) + 1),
\]
kjer je $P' \in f(U)$. Če to enakost seštejemo po okolicah vseh točk $P \in f^{-1}(X)$, dobimo
\[
m \leq \varphi(X) \leq \varphi(P').
\]

Pokažimo še, da so te množice zaprte v $\rs$. Naj bo $Q$ limita
zaporedja točk $Q_k \in \Sigma_m$, pri čemer je brez škode za
splošnost $b(P) = 0$ za vsak $P \in f^{-1}(Q_k)$. Ker imajo vse
množice $f^{-1}(Q_k)$ vsaj $m$ elementov, lahko najdemo tako
podzaporedje zaporedja $(Q_k)_{n=1}^\infty$, da lahko iz njihovih
praslik tvorimo $m$ konvergentnih zaporedij. Tako sledi
\[
\sum_{P \in f^{-1}(Q)} (b(P)+1) \geq m.
\]
Sledi, da so vse množice $\Sigma_m$ odprte in zaprte v $\rs$. Čim
je ena izmed množic $\Sigma_m$ neprazna, je tako enaka celotni
Riemannovi sferi, saj je ta povezana.
\end{proof}

Številu $m$ pravimo \emph{stopnja} preslikave $f$.

\begin{posledica}
Naj bo $M$ kompaktna Riemannova ploskev. Če je $f \colon M \to \C$
holomorfna preslikava, je konstantna.
\end{posledica}

\begin{proof}
Preslikavo $f$ lahko opazujemo kot preslikavo med Riemannovo
ploskvijo $M$ in Riemannovo sfero $\rs$. Če $f$ ni konstantna, ima
pozitivno stopnjo, kar pa ni mogoče, saj je $f^{-1}(\infty) = 0$.
\end{proof}

To posledico lahko pravzaprav dokažemo z uporabo lastnosti
holomorfnih preslikav. Ker je $M$ kompaktna namreč sledi, da je
taka tudi $f(M)$. Ker pa so nekonstantne holomorfne preslikave
odprte, pa sledi, da je $f(M)$ tudi odprta. To seveda pomeni, da je
$f(M) = \C$, kar je v protislovju s kompaktnostjo.

\begin{definicija}
Za kompaktni Riemannovo ploskvi $M$ in $N$ ter nekonstantno
preslikavo $f \colon M \to N$ definiramo
\emph{TOTAL BRANCHING NUMBER} kot
\[
B = \sum_{P \in M} b_f(P).
\]
\end{definicija}

Število je dobro definirano, saj je množica
$\setb{P \in M}{b_f(P)}$ diskretna in tako zaradi kompaktnosti
končna.

\begin{izrek}[Riemann-Hurwitz]
\label{iz:rie-hur}
Naj bosta $M$ in $N$ kompaktni Riemannovi ploskvi rodov $g$ in
$\gamma$, $f \colon M \to N$ pa nekonstantna preslikava stopnje
$n$. Tedaj za TOTAL BRANCHING NUMBER $B$ velja
\[
g = n(\gamma - 1) + 1 + \frac{B}{2}.
\]
\end{izrek}

\begin{proof}
Ker je množica $\setb{P \in M}{b_f(P) > 0}$ končna, jo lahko
uporabimo za triangulacijo ploskve $N$. Denimo, da ima
triangulacija $F$ lic, $E$ povezav in $V$ vozlišč. To
triangulacijo lahko z $f^{-1}$ preslikamo na $M$. Tako dobimo
triangulacijo ploskve $M$ z $nF$ lici, $nE$ povezavami in $nV - B$
vozlišči. Sledi, da je
\begin{align*}
F - E + V &= 2 - 2 \gamma,
\\
nF - nE + nV - B &= 2 - 2g.
\end{align*}
Iz teh enačb očitno sledi
\[
g = n(\gamma - 1) + 1 + \frac{B}{2}. \qedhere
\]
\end{proof}

\begin{definicija}
Naj bo $H \subseteq \Aut M$ končna podgrupa grupe avtomorfizmov
ploskve $M$. Na množici $\kvoc{M}{H}$ uvedemo kompleksno strukturo
na naslednji način:

\begin{enumerate}[i)]
\item Če je množica $H_P = \setb{h \in H}{h(P) = P}$ trivialna,
lokalna karta na dovolj majhni okolici $P$ inducira lokalno karto
pri $\pi(P)$.
\item Če je v lokalni koordinati $H_P$ generirana s preslikavo
$z \mapsto e^{\frac{2 \pi i}{k}} z$, za lokalno karto točke $P$
vzamemo $z^k$.
\end{enumerate}
\end{definicija}

Prepričamo se lahko, da so te lokalne karte med seboj kompatibilne.
Na kvocientu $\kvoc{M}{H}$ smo torej dobili kompleksno strukturo,
zato je to Riemannova ploskev.

\subsection{Riemann-Rochov izrek}

\begin{definicija}
\emph{Divizor} na Riemannovi ploskvi $M$ je formalni simbol
\[
\mathfrak{A} = \prod_{P \in M} P^{\alpha(P)},
\]
kjer za vsak $P$ velja $\alpha(P) \in \Z$ in je $\alpha(P) \ne 0$
za kvečjemu končno mnogo točk $P \in M$. \emph{Stopnja} divizorja
$\mathfrak{A}$ je definirana kot
\[
\deg \mathfrak{A} = \sum_{P \in M} \alpha(P).
\]
\end{definicija}

Divizorji na $M$ tvorijo grupo za naravno definirano množenje --
to grupo označimo z $\Div(M)$. Tako je
$\deg \colon \Div(M) \to \Z$ homomorfizem grup.

Za vsako neničelno meromorfno funkcijo $f \in \mathscr{K}(M)$
definiramo njen \emph{glavni divizor} kot\footnote{Glavne divizorje
na enak način definiramo še za meromorfne diferenciale.}
\[
(f) = \prod_{P \in M} P^{\ord_P f}.
\]
Definiramo lahko še \emph{divizor polov}
\[
f^{-1}(\infty) = \prod_{P \in M} P^{\max(-\ord_P f, 0)}
\]
in \emph{divizor ničel}
\[
f^{-1}(0) = \prod_{P \in M} P^{\max(\ord_P f, 0)}.
\]
Opazimo, da velja
\[
(f) = \frac{f^{-1}(0)}{f^{-1}(\infty)}.
\]
Na divizorjih uvedemo ekvivalenčno relacijo $\sim$ na naslednji način:
\[
\mathfrak{A} \sim \mathfrak{B} \iff
\mathfrak{A} \mathfrak{B}^{-1} = (f).
\]

\begin{lema}
Naj bo $M$ kompaktna Riemannova ploskev. Za vsako neničelno
funkcijo $f \in \mathscr{K}(M)$ velja
$\deg f^{-1}(0) = \deg f^{-1}(\infty)$. Ekvivalentno je
$\deg (f) = 0$.
\end{lema}

\begin{proof}
Stopnja divizorja polov funkcije $f$ je natanko število njenih
polov, štetih z večkratnostmi, stopnja divizorja ničel pa
število njenih ničel. Ti števili sta enaki po
trditvi~\ref{td:deg}.
\end{proof}

Posebej velja opomniti, da to pomeni, da imajo funkcije na
kompaktnih Riemannovih ploskvah enako število ničel in polov
(štetih z večkratnostmi).

Na divizorjih lahko uvedemo relacijo delne urejenosti kot
\[
\prod_{P \in M} P^{\alpha(P)} \geq \prod_{P \in M} P^{\beta(P)}
\iff
\forall P \in M \colon \alpha(P) \geq \beta(P).
\]
Pravimo, da je divizor $\mathfrak{A}$ \emph{efektiven}, če velja
$\mathfrak{A} \geq 1$. Ni težko videti, da je za vsak divizor
$\mathfrak{A}$ na $M$ množica
\[
L(\mathfrak{A}) =
\setb{f \in \mathscr{K}(M)}{(f) \geq \mathfrak{A}}
\]
vektorski prostor -- njegovo dimenzijo označimo z
$r(\mathfrak{A})$.

\begin{zgled}
Velja $r(1) = 1$. Pogoj $(f) \geq 1$ je namreč ekvivalenten temu,
da je $f$ holomorfna. Ker so vse holomorfne funkcije na kompaktnih
Riemannovih ploskvah konstantne, je zato $L(1) \cong \C$, kar je
enodimenzionalen prostor.
\end{zgled}

\begin{zgled}
Če je $\deg \mathfrak{A} > 0$, je $r(\mathfrak{A}) = 0$. Iz
neenakosti $(f) \geq \mathfrak{A}$ za neničenlno funkcijo $f$
namreč sledi $0 = \deg (f) \geq \deg \mathfrak{A} > 0$, kar je
protislovje.
\end{zgled}

Podobno je tudi
\[
\Omega(\mathfrak{A}) =
\setb{\omega}{\text{$\omega$ je meromorfen diferencial} \land
(\omega) \geq \mathfrak{A}}
\]
vektorski prostor. Označimo
$i(\mathfrak{A}) = \dim \Omega(\mathfrak{A})$.

\begin{trditev}\label{td:mero_dif}
Naj bo $\mathfrak{A}$ poljuben divizor in $\omega$ meromorfen
diferencial. Tedaj je
\[
i(\mathfrak{A}) = r \br{\mathfrak{A} (\omega)^{-1}}.
\]
\end{trditev}

\begin{proof}
Naj bo $\varphi \colon
\Omega(\mathfrak{A}) \to L \br{\mathfrak{A} (\omega)^{-1}}$
preslikava s predpisom
$\varphi \colon \zeta \mapsto \frac{\zeta}{\omega}$. Seveda je
predpis dobro definiran, ni pa težko videti, da je to izomorfizem
vektorskih prostorov. Sledi, da imata enako dimenzijo.
\end{proof}

\begin{izrek}[Riemann-Roch]
Naj bo $M$ kompaktna Riemannova ploskev roda $g$ in $\mathfrak{A}$
divizor na $M$. Tedaj velja
\[
r \br{\mathfrak{A}^{-1}} =
\deg \mathfrak{A} - g + 1 + i(\mathfrak{A}).
\]
\end{izrek}

Dokaz izreka najdemo v~\cite[theorem~III.4.11]{Farkas_Kra_1980}.

\begin{zgled}
Z uporabo zgornjega izreka lahko izračunamo $i(1)$. Velja namreč
\[
i(1) = r(1) - \deg 1 + g - 1 = g. \qedhere
\]
\end{zgled}

\begin{trditev}
Naj bo $\deg \mathfrak{A} > 2g - 2$. Tedaj je
$i(\mathfrak{A}) = 0$.
\end{trditev}

\begin{proof}
Naj bo $\omega \in i(1)$ neničelna holomorfen diferencial. Tedaj je
\[
r \br{(\omega)^{-1}} = \deg (\omega) - g + 1 + i \br{(\omega)}.
\]
Po trditvi~\ref{td:mero_dif} je $r \br{(\omega)^{-1}} = i(1) = g$
in $i \br{(\omega)} = r(1) = 1$. Od tod sledi, da je
$\deg (\omega) = 2g - 2$.

Sedaj dobimo
\[
i(\mathfrak{A}) = r \br{\mathfrak{A} (\omega)^{-1}} = 0,
\]
saj je $\deg \br{\mathfrak{A} (\omega)^{-1}} > 0$.
\end{proof}

\subsection{Weierstrassove točke}

\begin{izrek}[Weierstrass]
Naj bo $M$ ploskev roda $g > 0$ in $P \in M$. Tedaj obstaja
natanko $g$ števil
\[
1 = n_1 < n_2 < \dots < n_g < 2g,
\]
za katera ne obstaja funkcija $f \in \mathscr{K}(M)$, ki je
holomorfna na $M \setminus \set{P}$ in ima pol reda $n_j$ v $P$.
Tem številom pravimo GAP.
\end{izrek}

\begin{proof}
Najprej opazimo, da je število $n$ GAP natanko tedaj, ko je
$r(P^{-n}) = r(P^{1-n})$. Ker je $r(P^{-n}) \leq r(P^{1-n}) + 1$,
število $n$ ni GAP natanko tedaj, ko velja
\[
r \br{P^{-n}} - r \br{P^{1-n}} = 1.
\]
Po Riemann-Rochovem izreku velja
\[
r \br{P^{-k}} = k-g+1 + i \br{P^k},
\]
zato sledi
\begin{align*}
r \br{P^{-n}} - r(1) &=
\sum_{k=1}^n \br{r \br{P^{-k}} - r \br{P^{1-k}}}
\\
&=
\sum_{k=1}^n \br{1 + i \br{P^k} - i \br{P^{k-1}}}
\\
&=
n + i \br{P^n} - i(1).
\end{align*}
Ker je $i(1) = g$ in za vse $n > 2g-2$ velja $i(P^n) = 0$, sledi
\[
r \br{P^{-n}} - 1 = n - g.
\]
Opazimo, da leva stran šteje ravno število NEGAPOV $\leq n$.
Sledi, da je GAPOV natanko $g$ in so vsi strogo manjši od $2g$.
\end{proof}

Izkaže se, da je lažje analizirati komplement tega zaporedja, torej
števila
\[
1 < \alpha_1 < \dots < \alpha_g = 2g,
\]
za katera obstaja funkcija s polom reda $\alpha_j$ v $P$. Če sta
števili $\alpha_i$ in $\alpha_j$ NEGAPA, je tako tudi število
$\alpha_i + \alpha_j$, saj lahko vzamemo kar produkt pripadajočih
funkcij. Posebej je vsak večkratnik NEGAPA spet NEGAP. Če je
$\alpha_1 = 2$, so tako vsa soda števila NEGAPI in GAPI natanko
liha števila, manjša od $2g$.

\begin{lema}
Za vsako naravno število $j < g$ velja
\[
\alpha_j + \alpha_{g-j} \geq 2g.
\]
\end{lema}

\begin{proof}
Denimo, da je $\alpha_j + \alpha_{g-j} < 2g$. Tedaj so vsa števila
$\alpha_k + \alpha_{g-j}$ za $k \leq j$ NEGAPI, manjši od $2g$.
Tako imamo skupaj vsaj $g-j + j + 1 = g+1$ NEGAPOV, manjših od
$2g$. To je seveda protislovje.
\end{proof}

\begin{lema}
Velja neenakost
\[
\sum_{j=1}^{g} \alpha_j \geq g \cdot(g+1)
\]
z enakostjo natanko tedaj, ko je $\alpha_1 = 2$.
\end{lema}

\begin{proof}
Za dokaz neenakosti je dovolj uporabiti prejšnjo lemo. Zgornja
izraza sta enaka natanko tedaj, ko za vsak $j < g$ velja
\[
\alpha_j + \alpha_{g-j} = 2g.
\]
Če je število $\alpha$ NEGAP, je tako torej tudi $2g - \alpha$.
Opazimo, da je za NEGAPA $\alpha_i < \alpha_j$ tudi
$\alpha_i + (2g - \alpha_j)$ NEGAP, zato je tak tudi
\[
2g - (\alpha_i + (2g - \alpha_j)) = \alpha_j - \alpha_i.
\]
Sledi, da je tudi vsaka razlika NEGAPOV NEGAP. Sledi, da so vsi
NEGAPI večkratnik najmanjšega NEGAPA (osnovni izrek o deljenju),
kar pa takoj implicira $\alpha_1 = 2$.
\end{proof}

Število $n$ je GAP natanko tedaj, ko je
$r \br{P^{-n}} = r \br{P^{1-n}}$. To je po Riemann-Rochovem izreku
ekvivalentno $i \br{P^{n-1}} - i \br{P^n} = 1$. Sledi, da imajo
holomorfni diferenciali na $M$ v točki $P$ lahko red enak le enemu
iz števil
\[
n_1 - 1, n_2 - 1, \dots, n_g - 1.
\]
Posebej, obstaja baza $\setb{\omega_i}{1 \leq i \leq g}$
holomorfnih diferencialov, pri čemer velja
$\ord_P \omega_i = n_i - 1$.

\begin{definicija}
TEŽA točke $P \in M$ je vsota
\[
\tau(P) = \sum_{j=1}^g (n_j - j),
\]
kjer so $n_j$ GAPI za $P$.
\end{definicija}

\begin{lema}
Naj bodo $\varphi_1$, $\varphi_2$, \dots, $\varphi_n$ meromorfne
preslikave s paroma različnim redom v točki $P \in X$. Tedaj za
determinanto
\[
\Phi(z) = \det
\begin{bmatrix}
\varphi_1(z)    & \varphi_2(z)  & \dots  & \varphi_n(z)  \\
\varphi_1'(z)   & \varphi_2'(z) & \dots  & \varphi_n'(z) \\
\vdots          & \vdots        & \ddots & \vdots        \\
\varphi_1^{(n-1)}(z)            & \varphi_2^{(n-1)}      &
\dots                           & \varphi_n^{(n-1)}(z)
\end{bmatrix}
\]
velja
\[
\ord_z \Phi = \sum_{i=1}^n \br{\ord_z \varphi_i - i + 1}.
\]
\end{lema}

\begin{proof}
Za lažji zapis pišimo
\[
\Phi(z) = \det
\begin{bmatrix}
\varphi_1(z) & \dots & \varphi_n(z)
\end{bmatrix}.
\]
Trditev dokažemo z indukcijo -- trditev očitno velja za $n=1$. Za
indukcijski korak si bomo pomagali z dejstvom, da za vsako
meromorfno funkcijo $f$ velja
\[
\Phi_f = \det
\begin{bmatrix}
f \cdot \varphi_1(z) & \dots & f \cdot \varphi_n(z)
\end{bmatrix}
=
f^n \cdot \det
\begin{bmatrix}
\varphi_1(z) & \dots & \varphi_n(z)
\end{bmatrix}.
\]
Razpišemo lahko namreč
\[
\Phi_f
=
\det
\begin{bmatrix}
f  \varphi_1 & \dots & f \varphi_n \\
f \varphi_1' + f' \varphi_1 & \dots & f \varphi_n' + f' \varphi_n \\
\vdots & & \vdots \\
f \varphi_1^{(n-1)} + \dots + f^{(n-1)} \varphi_1 & \dots & f \varphi_n^{(n-1)} + \dots + f^{(n-1)} \varphi_n
\end{bmatrix}.
\]
S preprostimi linearnimi kombinacijami lahko sedaj dobimo
\[
\Phi_f = \det
\begin{bmatrix}
f \varphi_1         & \dots & f \varphi_n         \\
f \varphi_1'        & \dots & f \varphi_n'        \\
\vdots              &       & \vdots              \\
f \varphi_1^{(n-1)} & \dots & f \varphi_n^{(n-1)}
\end{bmatrix}
= f^n \Phi.
\]
Tako dobimo
\[
\det
\begin{bmatrix}
\varphi_1 & \dots \varphi_n
\end{bmatrix}
=
\varphi_1^n \cdot \det
\begin{bmatrix}
1 & \frac{\varphi_2}{\varphi_1} &
\dots & \frac{\varphi_n}{\varphi_1}
\end{bmatrix}
=
\varphi_1^n \cdot \det
\begin{bmatrix}
\br{\frac{\varphi_2}{\varphi_1}}' & \dots &
\br{\frac{\varphi_n}{\varphi_1}}'
\end{bmatrix}.
\]
Ker za vsak $i$ velja $\ord_z \varphi_1 \ne \ord_z \varphi_i$,
sledi
\[
\ord_z \br{\frac{\varphi_i}{\varphi_1}}' =
\ord_z \varphi_i - \ord_z \varphi_1 - 1.
\]
To so paroma različna števila, zato lahko uporabimo indukcijsko
predpostavko. Dobimo
\begin{align*}
\ord_z \Phi &=
n \cdot \ord_z \varphi_1 +
\sum_{i=2}^n \br{\ord_z \varphi_i - \ord_z \varphi_1 - 1 - (i - 2)}
\\
&=
\ord_z \varphi_1 + \sum_{i=2}^n \br{\ord_z \varphi_i - i + 1}
\\
&=
\sum_{i=1}^n \br{\ord_z \varphi_i - i + 1}. \qedhere
\end{align*}
\end{proof}

Posledično lahko zapišemo $\tau(P) = \ord_P \Phi$, pri čemer za
$\varphi_i$ vzamemo kar $\omega_i$.

\begin{trditev}
Naj bo $M$ kompaktna Riemannova ploskev z rodom $g \geq 2$. Tedaj
je
\[
\sum_{P \in M} \tau(P) = g^3-g.
\]
\end{trditev}

\begin{proof}
Pokažimo, da je zgoraj definiran $\Phi$ holomorfen $m$-diferencial
za $m = \frac{g \cdot (g+1)}{2}$. Denimo, da $\omega_i$ priredi
okolici $U$ karto $\varphi$, okolici $V$ pa karto $\psi$, $f$ pa
naj bo prehodna preslikava. Tako velja
\[
\psi(f(z)) f'(z) = \varphi(z),
\]
dokazujemo pa
\[
f'(z)^m \cdot \det
\begin{bmatrix}
\psi_1 & \dots & \psi_g
\end{bmatrix}
=
\det
\begin{bmatrix}
\varphi_1 & \dots & \varphi_g
\end{bmatrix}.
\]
Velja pa
\begin{align*}
\det
\begin{bmatrix}
\varphi_1 & \dots & \varphi_g
\end{bmatrix}
&=
\det
\begin{bmatrix}
(\psi_1 \circ f) \cdot (f') & \dots & (\psi_g \circ f) \cdot (f')
\end{bmatrix}
\\
&=
(f')^g \cdot \det
\begin{bmatrix}
\psi_1 \circ f & \dots & \psi_g \circ f
\end{bmatrix}.
\end{align*}
Po pravilu odvoda kompozituma lahko iz vrstice $i$ izpostavimo še
$(f')^{i-1}$. Tako dobimo
\[
\det
\begin{bmatrix}
\varphi_1 & \dots & \varphi_g
\end{bmatrix}
=
(f')^m \cdot \br{\det
\begin{bmatrix}
\psi_1 & \dots & \psi_g
\end{bmatrix}
\circ f}.
\]
Spomnimo se, da za meromorfen diferencial $\omega$ velja
$\deg (\omega) = 2g - 2$. Ker je $\frac{\omega^m}{\Phi}$ meromorfna
funkcija, sledi
\[
\deg (\Phi) = m \cdot \deg (\omega) = m \cdot (2g - 2).
\]
Tako je
\[
\sum_{P \in M} \tau(P) =
\sum_{P \in M} \ord_P \Phi =
(g-1) \cdot g \cdot (g+1). \qedhere
\]
\end{proof}

\begin{definicija}
Točka $P \in M$ je \emph{Weierstrassova točka}, če na $M$ obstaja
neničelna holomorfen diferencial z ničlo reda vsaj $g$ v
$P$.\footnote{V splošnem definiramo $q$-Weierstrassove točke --
obstaja $q$-diferencial z ničlo reda vsaj $\dim \mathscr{H}^q(M)$.}
\end{definicija}

\begin{trditev}
Ekvivalentno, vsaj eno izmed števil $2, \dots, g$ ni GAP.
\end{trditev}

\begin{proof}
Obstoj diferencialne $1$-forme z ničlo reda vsaj $g$ v $P$ je
ekvivalentna pogoju $i(P^g) > 0$. Po Riemann-Rochovem izreku je ta
neenakost ekvivalentna
\[
r \br{P^{-g}} - 1 > 0,
\]
oziroma $r \br{P^{-g}} \geq 2$. Ker je $r(1) = 1$, med
$2, \dots, g$ obstaja število, ki ni GAP.
\end{proof}

\begin{trditev}
Naj bo $M$ kompaktna Riemannova ploskev roda $g \geq 2$. Tedaj za
število $w$ Weierstrassovih točk veljata oceni
\[
2g + 2 \leq w \leq g^3 - g.
\]
\end{trditev}

\begin{proof}
Ker je $\tau(P) \geq 1$ za vsako Weierstrassovo točko in velja
\[
\sum_{P \in M} \tau(P) = g^3 - g,
\]
takoj sledi $w \leq g^3 - g$. Velja pa
\begin{align*}
\tau(P) &= \sum_{j=1}^g (n_j - j)
\\
&=
\sum_{j=1}^{2g} j - \sum_{j=1}^g \alpha_j - \sum_{j=1}^g j
\\
&=
g(2g+1) - \frac{g(g+1)}{2} - \sum_{j=1}^{g-1} \alpha_j
\\
&\leq
g(2g+1) - \frac{g(g+1)}{2} - g(g+1)
\\
&=
\frac{g(g-1)}{2}.
\end{align*}
Posledično je res $w \geq 2g + 2$.
\end{proof}

\subsection{Hipereliptične ploskve}

\begin{definicija}
Kompaktna Riemannova ploskev $M$ je \emph{hipereliptična}, če
obstaja nekonstantna meromorfna funkcija $f \colon M \to \rs$ z
natanko dvema poloma.\footnote{Pri tem pole štejemo z
večkratnostmi.}
\end{definicija}

Ekvivalentno to pomeni, da obstaja tak efektiven divizor
$D \in \Div M$, da je $\deg D = 2$ in $r(D^{-1}) \geq 2$.

\begin{trditev}
Vsaka kompaktna Riemannova ploskev roda $g \leq 2$ je
hipereliptična.
\end{trditev}

\begin{trditev}
Weierstrassove ploskve imajo natanko $2g+2$ BRANCH točk.
\end{trditev}

\begin{proof}
Po izreku~\ref{iz:rie-hur} velja
\[
g = 2 \cdot (0-1) + 1 + \frac{B}{2}. \qedhere
\]
\end{proof}

\begin{trditev}
BRANCH točke preslikave $f$ so natanko Weierstrassove točke ploskve
$M$.
\end{trditev}

\begin{proof}
Naj bo $P \in M$ BRANCH točka. Če je $P$ pol funkcije $f$, je
njegova stopnja tako enaka $2$. V nasprotnem primeru ima funkcija
\[
g \equiv \frac{1}{f - f(P)}
\]
pol stopnje $2$ v $P$. V obeh primerih sledi, da $2$ ni GAP za
točko $P$, zato je ta Weierstrassova.

Vsaka BRANCH točka ima tako TEŽO
\[
\sum_{k=1}^g \br{2k-1} - \sum_{k=1}^g k = \frac{1}{2} g (g-1),
\]
zato je njihova skupna teža $g^3 - g$. Sledi, da so to vse
Weierstrassove točke.
\end{proof}

\begin{lema}
Naj bo $P$ Weierstrassova točka hipereliptične ploskve $M$ in
$f \in \mathscr{K}(M)$ funkcija stopnje $2$. Tedaj velja
$f^{-1}(\infty) \sim P^2$.
\end{lema}

\begin{proof}
Točka $P$ je BRANCH točka funkcije $f$. Če je $P$ pol te funkcije,
je zato reda $2$ in je $f^{-1}(\infty) = P^2$. V nasprotnem primeru
definiramo funkcijo
\[
g = \frac{1}{f - f(P)}.
\]
Ni težko videti, da je $(g) = f^{-1}(\infty) P^{-2}$.
\end{proof}

\begin{trditev}
Naj bosta $f$ in $g$ dve funkciji $f \colon M \to \rs$ stopnje
$2$. Tedaj obstaja Möbiusova transformacija $A$, za katero je
\[
g = A \circ f.
\]
\end{trditev}

\begin{proof}
Naj bo $f^{-1}(\infty) = P_1 Q_1$ in $g^{-1}(\infty) = P_2 Q_2$.
Ker na $M$ ne obstajajo funkcije stopnje $1$, sledi
$r(P_1^{-1} Q_1^{-1}) = r(P_2^{-1} Q_2^{-1}) = 2$. Prostora
$L(P_1^{-1} Q_1^{-1})$ in $L(P_2^{-1} Q_2^{-1})$ imata tako
zaporedoma bazi $\set{1, f}$ in $\set{1, g}$. Ker za Weierstrassovo
točko $P$ velja $P_1 Q_1 \sim P^2 \sim P_2 Q_2$, sledi, da obstaja
meromorfna preslikava $h$, za katero je
$(h) = P_1 Q_1 P_2^{-1} Q_2^{-1}$. Ker je s predpisom
$\varphi \mapsto h \cdot \varphi$ očitno podan izomorfizem
prostorov $L(P_1^{-1} Q_1^{-1})$ in $L(P_2^{-1} Q_2^{-1})$,
obstajajo konstante $\alpha, \beta, \gamma$ in $\delta$, za katere
je
\[
1 = \alpha h + \beta hf
\quad \text{in} \quad
g = \gamma h + \delta hf.
\]
Tako lahko izrazimo
\[
g = \frac{\gamma + \delta f}{\alpha + \beta f}. \qedhere
\]
\end{proof}

\begin{trditev}
Naj bo $M$ kompaktna Riemannova ploskev roda $g$. Tedaj je $M$
hipereliptična natanko tedaj, ko obstaja involucija $J \in \Aut M$
z natanko $2g + 2$ fiksnimi točkami.
\end{trditev}

\begin{proof}
Predpostavimo najprej, da je $M$ hipereliptična. Naj bo
$f \colon M \to \rs$ meromorfna funkcija stopnje $2$. Za vsak
$P \in M$ tako obstaja še ena točka $Q \in M$, za katero je
$f(P) = f(Q)$ (če je $\ord_P f = 2$, vzamemo $Q=P$). Tako lahko
enostavno definiramo $J(P) = Q$. Ni težko videti, da je $J$
res involucija z $2g + 2$ fiksnimi točkami.

Če je $Q = J(P) \ne P$, lahko na okolici $U_Q$ točke $Q$ zapišemo
\[
J(X) = \br{\eval{f}{U_Q}{}}^{-1}(f(X)),
\]
zato je $J$ holomorfna na $M \setminus W$. Če pa je $J(P) = P$, pa
je $h = \sqrt{f - f(P)}$ lokalna koordinata, za katero velja
$J(h) = -h$, saj je
\[
f(P_h) = h^2 + f(P) = (-h)^2 + f(P) = f(P_{-h}).
\]
Tako je $J$ holomorfna tudi na $W$.

Predpostavimo sedaj, da obstaja involucija $J \in \Aut M$ z
$2g + 2$ fiksnimi točkami. Ker se projekcija
$f \colon M \to \kvoc{M}{\skl{J}}$ BRANCHA v natanko $2g + 2$
točkah, po izreku~\ref{iz:rie-hur} sledi, da je rod ploskve
$\kvoc{M}{\skl{J}}$ enak $0$. Sledi, da je
$\kvoc{M}{\skl{J}} \cong \rs$, zato je $f$ meromorfna funkcija z
dvema poloma.
\end{proof}

Opazimo, da so fiksne točke hipereliptične involucije natanko
Weierstrassove točke.

\begin{trditev}
Naj bo $M$ hipereliptična Riemannova ploskev roda $g \geq 2$ in
$T \in \Aut M$. Če je $T \not \in \skl{J}$, ima $T$ kvečjemu $4$
fiksne točke.
\end{trditev}

\begin{proof}
Naj bo $f \colon M \to \rs$ funkcija z natanko dvema poloma. Tedaj
je taka tudi $f \circ T$, zato obstaja Möbiusova transformacija
$A$, za katero je
\[
f \circ T = A \circ f.
\]
Naj bo $P$ fiksna točka avtomorfizma $T$. Sledi, da je
\[
A(f(P)) = f(T(P)) = f(P),
\]
zato je $f(P)$ fiksna točka preslikave $A$. Opazimo, da je
$A \ne \id$, saj bi v nasprotnem primeru veljalo $f \circ T = f$,
kar implicira $T \in \skl{J}$. Tako ima $A$ kvečjemu $2$ fiksni
točki, zato jih ima $T$ največ $4$.
\end{proof}
