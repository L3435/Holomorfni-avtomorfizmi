\section{Riemannove ploskve}

\subsection{Gladke in kompleksne mnogoterosti}

%TODO Definicije gladkih mnogoterosti

\subsection{Riemann-Rochov izrek}

\begin{definicija}
\emph{Delitelj} na Riemannovi ploskvi $M$ je formalni simbol
\[
\mathfrak{A} = \prod_{P \in M} P^{\alpha(P)},
\]
kjer za vsak $P$ velja $\alpha(P) \in \Z$ in je $\alpha(P) \ne 0$
za kvečjemu končno mnogo točk $P \in M$. \emph{Stopnja} delitelja
$\mathfrak{A}$ je definirana kot
\[
\deg \mathfrak{A} = \sum_{P \in M} \alpha(P).
\]
\end{definicija}

Delitelji na $M$ tvorijo grupo za naravno definirano množenje --
to grupo označimo z $\Div(M)$. Tako je
$\deg \colon \Div(M) \to \Z$ homomorfizem grup.

Za vsako neničelno meromorfno funkcijo $f \in \mathscr{K}(M)$
definiramo njen \emph{glavni delitelj} kot
\[
(f) = \prod_{P \in M} P^{\ord_P f}.
\]
Definiramo lahko še \emph{polarni delitelj}
\[
f^{-1}(\infty) = \prod_{P \in M} P^{\max(-\ord_P f, 0)}
\]
in \emph{ničelni delitelj}
\[
f^{-1}(0) = \prod_{P \in M} P^{\max(\ord_P f, 0)}.
\]
Opazimo, da velja
\[
(f) = \frac{f^{-1}(0)}{f^{-1}(\infty)}.
\]

\begin{lema}
Za vsako neničelno funkcijo $f \in \mathscr{K}(M)$ velja
$\deg (f) = 0$. Posledično je
$\deg f^{-1}(0) = \deg f^{-1}(\infty)$.
\end{lema}

%TODO Weights
%TODO Riemann-Roch

\subsection{Weierstrassove točke}

\begin{izrek}
Naj bo $M$ ploskev roda $g > 0$ in $P \in M$. Tedaj obstaja
natanko $g$ števil
\[
1 = n_1 < n_2 < \dots < n_g < 2g,
\]
za katera ne obstaja funkcija $f \in \mathscr{K}(M)$, ki je
holomorfna na $M \setminus \set{P}$ in ima pol reda $n_j$ v $P$.
Tem številom pravimo GAP.
\end{izrek}

\begin{definicija}
Točka $P \in M$ je \emph{Weierstrassova točka}, če na $M$ obstaja
neničelna holomorfna diferencialna $1$-forma z ničlo reda vsaj
$g$ v $P$.\footnote{V splošnem definiramo $q$-Weierstrassove točke
-- obstaja $q$-forma z ničlo reda vsaj $\dim \mathscr{H}^q(M)$.}
\end{definicija}

\begin{lema}
Ekvivalentno, vsaj eno izmed števil $2, \dots, g$ ni GAP.
\end{lema}

\begin{proof}
Obstoj diferencialne $1$-forme z ničlo reda vsaj $g$ v $P$ je
ekvivalentna pogoju $i(P^g) > 0$. Po Riemann-Rochovem izreku je ta
neenakost ekvivalentna
\[
r \br{P^{-g}} - 1 > 0,
\]
oziroma $r \br{P^{-g}} \geq 2$. Ker je $r(1) = 1$, med
$2, \dots, g$ obstaja število, ki ni GAP.
%TODO Zakaj r(1) = 1?
\end{proof}

\begin{lema}
Naj bo $M$ kompaktna Riemannova ploskev roda $g \geq 2$. Tedaj za
število $w$ Weierstrassovih točk veljata oceni
\[
2g + 2 \leq w \leq g^3 - g.
\]
\end{lema}

\subsection{Hipereliptične ploskve}
