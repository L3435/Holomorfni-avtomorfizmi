%\documentclass[mat1, tisk]{fmfdelo}
\documentclass[mat1]{fmfdelo}

\avtor{Luka Horjak}

\naslov{Holomorfni avtomorfizmi}

\title{Holomorphic automorphisms}

\mentor{prof.~dr.~Miran Černe}

\letnica{2023} 

% - povzetek v slovenščini
%   V povzetku na kratko opišite vsebinske rezultate dela. Sem ne sodi razlaga
%   organizacije dela, torej v katerem razdelku je kaj, pač pa le opis vsebine.
\povzetek{...}

% - povzetek v angleščini
\abstract{...}

% - klasifikacijske oznake, ločene z vejicami
%   Oznake, ki opisujejo področje dela, so dostopne na strani https://www.ams.org/msc/
\klasifikacija{30F10, 30C20}

% - ključne besede, ki nastopajo v delu, ločene s \sep
\kljucnebesede{...\sep ...}

% - angleški prevod ključnih besed
\keywords{...\sep ...} % angleški prevod ključnih besed

% - angleško-slovenski slovar strokovnih izrazov
% - manually sorted
\slovar{
\geslo{Riemann surface}{Riemannova ploskev}
}

% - ime datoteke z viri (vključno s končnico .bib), če uporabljate BibTeX

\usepackage{Diploma}
\literatura{Diploma.bib}
\nocite{*} %TODO To odstrani

\begin{document}

%\section{Uvod}
% ...
\section{Holomorfni avtomorfizmi v kompleksni ravnini}

\subsection{Enostavno povezana območja}

Da lahko govorimo o holomorfnih funkcijah, se bomo omejili na
odprte podmnožice kompleksne ravnine.

\begin{definicija}
\emph{Holomorfen avtomorfizem} odprte množice $\Omega \subseteq \C$
je bijektivna holomorfna preslikava $f \colon \Omega \to \Omega$ s
holomorfnim inverzom.
\end{definicija}

Znano je, da je zadosten pogoj že to, da je $f$ bijektivna in
holomorfna. Opazimo še, da množica avtomorfizmov nekega območja
tvori grupo z operacijo kompozitum. To grupo označimo z
$\Aut(\Omega)$.

Prepričamo se lahko, da lahko avtomorfizme nepovezanih množic
opišemo z avtomorfizmi, ki komponente med seboj permutirajo. V
nadaljevanju se bomo tako omejili na povezane množice.

\begin{definicija}
\emph{Območje} v kompleksni ravnini $\C$ je vsaka odprta povezana
množica.
\end{definicija}

Grupe avtomorfizmov nekaterih območij so splošno znane -- za
kompleksno ravnino in enotski disk ju dobimo s pomočjo
Liouvillovega izreka in Schwarzove leme:

\begin{zgled}
Grupa avtomorfizmov kompleksne ravnine je enaka
\[
\Aut(\C) = \setb{z \mapsto az + b}{a \ne 0}. \qedhere
\]
\end{zgled}

\begin{zgled}
Naj bo $\dsk$ odprt enotski disk v $\C$. Tedaj je
\[
\Aut(\dsk) =
\setb{z \mapsto e^{i \theta} \cdot \frac{z-a}{1 - \oline{a} z}}
{a \in \dsk \land \theta \in [0, 2\pi)}. \qedhere
\]
\end{zgled}

\begin{figure}[!ht]
\centering

\begin{asy}
int ITER = 1000;
real STEP = 0.3;
pair a = (0.6,-0.3);
real theta = 0.3pi;
pair aut(pair X){
  return X;
}

path Re(real y, pair A(pair)){
  guide g;
  for(int i = 0; i < 1000; ++i){
    g = g--(A((2*sqrt(1-y*y)*i/ITER - sqrt(1-y*y),y)));
  }
  path p = g;
  return p;
}

path Im(real x, pair A(pair)){
  guide g;
  for(int i = 0; i < 1000; ++i){
    g = g--(A((x,2*sqrt(1-x*x)*i/ITER - sqrt(1-x*x))));
  }
  path p = g;
  return p;
}
draw(circle((0,0),1));
for(real t = -0.9; t <= 1; t += STEP){
	draw(Re(t, aut),red);
	draw(Im(t, aut),blue);
}
dot(a);
xaxis("$\Re$",Ticks(Label(fontsize(8pt)),new real[]{}),xmin=-1.25,xmax=1.25);
yaxis("$\Im$",Ticks(Label(fontsize(8pt)),new real[]{}),ymin=-1.25,ymax=1.25);
\end{asy}
\hfill
\begin{asy}
int ITER = 1000;
real STEP = 0.3;
pair a = (0.6,-0.3);
real theta = 0.3pi;
pair aut(pair X){
  return exp(I*theta) * (X-a) / (1-conj(a)*X);
}

path Re(real y, pair A(pair)){
  guide g;
  for(int i = 0; i < 1000; ++i){
    g = g--(A((2*sqrt(1-y*y)*i/ITER - sqrt(1-y*y),y)));
  }
  path p = g;
  return p;
}

path Im(real x, pair A(pair)){
  guide g;
  for(int i = 0; i < 1000; ++i){
    g = g--(A((x,2*sqrt(1-x*x)*i/ITER - sqrt(1-x*x))));
  }
  path p = g;
  return p;
}
draw(circle((0,0),1));
for(real t = -0.9; t <= 1; t += STEP){
	draw(Re(t, aut),red);
	draw(Im(t, aut),blue);
}
dot(aut((0,0)));
xaxis("$\Re$",Ticks(Label(fontsize(8pt)),new real[]{}),xmin=-1.25,xmax=1.25);
yaxis("$\Im$",Ticks(Label(fontsize(8pt)),new real[]{}),ymin=-1.25,ymax=1.25);
\end{asy}
\caption{Primer avtomorfizma diska z označenima točkama $f^{-1}(0)$
in $f(0)$}
\end{figure}

Izkaže se, da smo s tem do izomorfizma natančno opisali grupe
avtomorfizmov vseh povezanih in enostavno povezanih množic v $\C$.
Velja namreč naslednja lema:

\begin{lema}
Naj bosta $\Omega_1$ in $\Omega_2$ biholomorfno ekvivalentni
območji v $\C$. Tedaj je $\Aut(\Omega_1) \cong \Aut(\Omega_2)$.
\end{lema}

\begin{proof}
Naj bo $f \colon \Omega_1 \to \Omega_2$ biholomorfna preslikava.
Sedaj definiramo preslikavo
$\Phi \colon \Aut(\Omega_1) \to \Aut(\Omega_2)$ s predpisom
\[
\Phi(\phi) = f^{-1} \circ \phi \circ f.
\]
Ker je s predpisom
\[
\Phi^{-1}(\psi) = f \circ \psi \circ f^{-1}
\]
očitno podan predpis inverza preslikave $\Phi$, je ta bijektivna.
Velja pa
\[
\Phi(\phi \circ \psi) = f^{-1} \circ \phi \circ \psi \circ f =
\br{f^{-1} \circ \phi \circ f} \circ
\br{f^{-1} \circ \psi \circ f} =
\Phi(\phi) \circ \Phi(\psi),
\]
zato je $\Phi$ izomorfizem.
\end{proof}

Spomnimo se na Riemannov upodobitveni izrek, ki pravi, da je
vsako povezano in enostavno povezano območje v kompleksni ravnini
ali biholomorfno ekvivalentno $\dsk$ ali pa kar enako $\C$. Grupe
avtomorfizmov povezanih in enostavno povezanih območij so do
izomorfizma natančno tako le $\Aut(\dsk)$ in $\Aut(\C)$.

Omenimo še, da lahko kompleksno ravnino dopolnimo do Riemannove
sfere $\rs$. Znano je, da so njeni avtomorfizmi Möbiusove
transformacije, torej
\[
\Aut \br{\rs} =
\setb{z \mapsto \frac{az + b}{cz + d}}{ad - bc = 1}.
\]

\subsection{Punktirani diski in kolobarji}

Razdelek je povzet po~\cite{cerne}.

Po obravnavi enostavno povezanih območij so naslednji korak
območja z ">luknjami"<. Eden izmed osnovnejših primerov takih
območij je punktirani disk
$\dsk_\alpha = \dsk \setminus \set{\alpha}$.

Disk $\dskx = \dsk \setminus \set{0}$ je biholomorfno
ekvivalenten vsakemu punktiranemu disku, saj je preslikava
$f \colon \dsk_\alpha \to \dskx$ s predpisom
\[
f(z) = \frac{z - \alpha}{1 - \oline{\alpha} z}
\]
biholomorfna. Sledi, da je
$\Aut(\dsk_\alpha) \cong \Aut(\dskx)$.

\begin{trditev}
Za punktirani disk velja
\[
\Aut(\dskx) =
\setb{z \mapsto e^{i \pi \theta} z}{\theta \in [0, 2\pi)}.
\]
\end{trditev}

\begin{proof}
Naj bo $f \colon \dskx \to \dskx$ poljuben avtomorfizem. Tedaj je
točka $0$ izolirana singularnost funkcije $f$. Ker je $f$ omejena,
je to odpravljiva singularnost.

Naj bo $\widetilde{f} \colon \dsk \to \C$ funkcija, ki jo dobimo
tako, da $f$ razširimo na celoten disk. Predpostavimo, da velja
$\widetilde{f}(0) \ne 0$. Ker je $\widetilde{f}$ holomorfna in
nekonstantna, je odprta, zato sledi $\abs{\widetilde{f}(0)} < 1$.
Oglejmo si točko $\alpha \ne 0$, za katero je
$f(\alpha) = \widetilde{f}(0)$. Izberemo si lahko disjunktni
okolici $U$ in $V$ točk $0$ in $\alpha$. Ker je $\widetilde{f}$
odprta, je odprta tudi množica
$W = \widetilde{f}(U) \cap \widetilde{f}(V)$. Hitro opazimo, da
velja $\widetilde{f}(0) \in W$, zato je ta množica neprazna.
Sledi, da je $W$ neskončna, kar je protislovje, saj velja
$f(U \setminus \set{0}) \cap f(V) = \emptyset$.

Sledi, da je $\widetilde{f}(0) = 0$, zato je $\widetilde{f}$
avtomorfizem diska. Tako dobimo inkluzijo
\[
\Aut(\dskx) \subseteq \setb{f \in \Aut(\dsk)}{f(0) = 0}.
\]
Ni težko preveriti, da velja tudi obratna inkluzija, oziroma
\[
\Aut(\dskx) =
\setb{z \mapsto e^{i \pi \theta} z}{\theta \in [0, 2\pi)}.
\qedhere
\]
\end{proof}

Kaj pa se zgodi, če število lukenj povečamo? Naj bo
$\dsk_2 = \dsk \setminus \set{0, \alpha}$.\footnote{Podobno kot
prej lahko privzamemo, da je ena izmed lukenj enaka $0$.} Po
enakem razmisleku kot prej ugotovimo, da za vsak avtomorfizem
$f \in \Aut(\dsk_2)$ velja $f(0) \in \dsk$ in hkrati
$f(0) \not \in \dsk_2$. Enako velja za točko $\alpha$. Sedaj ni
težko videti, da velja
\[
\Aut(\dsk_2) =
\skl{z \mapsto \frac{\alpha - z}{1 - \oline{\alpha} z}} \cong
\Z_2,
\]
saj je avtomorfizem diska natančno določen z dvema točkama.

Sedaj si oglejmo še kolobar $R = \dsk \setminus \oline{\dsk(r)}$.
Naj bo $f \colon R \to R$ avtomorfizem. Po Carathéodoryjevem
izreku~\cite[izrek~5.1.1 in opomba~5.1.2]{krantz} se $f$ zvezno
razširi na $\partial R$. Ker lahko $f$ komponiramo s preslikavo
$\varphi \colon z \mapsto \frac{\rho}{z}$, lahko predpostavimo, da
je $f(\partial \dsk) = \partial \dsk$.

Naj bo
\[
u(z) = \log \abs{f(z)} - \log \abs{z}.
\]
Ker je logaritem absolutne vrednosti holomorfne funkcije brez ničel
harmonična funkcija, je $\lapl u = 0$. Ker je
$\eval{u}{\partial R}{} = 0$, je po principu maksima $u = 0$. Tako
sledi $\abs{f(z)} = \abs{z}$ in
\[
\abs{\frac{f(z)}{z}} = 1.
\]
Ker je $\frac{f(z)}{z}$ holomorfna in so nekonstantne holomorfne
preslikave odprte, sledi, da je $\frac{f(z)}{z}$ konstantna. Tako
so vsi avtomorfizmi kolobarja kar $z \mapsto e^{i \theta} z$ in
$z \mapsto e^{i \theta} \frac{r}{z}$.

\begin{figure}[!ht]
\centering

\begin{asy}
int ITER = 1000;
real STEP = 0.26;
pair a = (0.6,-0.3);
real theta = 0.3pi;
real r = 0.4;

pair aut(pair X){
  return X;
}

path Re(real y, pair A(pair), bool first = true){
  guide g;
  if(y * y >= r * r) {
    if(!first) return (0,0);
    real d = 2 * sqrt(1-y*y);
    real start = -sqrt(1-y*y);
    for(int i = 0; i < 1000; ++i){
      g = g--(A((d*i/ITER + start, y)));
    }	
  }
  else {
    real d = sqrt(1-y*y) - sqrt(r*r-y*y);
    real start = sqrt(r*r-y*y);
    if(!first) start = -sqrt(1-y*y);
    for(int i = 0; i < 1000; ++i){
      g = g--(A((d*i/ITER + start, y)));
    }
  }
  path p = g;
  return p;
}

path Im(real x, pair A(pair), bool first = true){
  guide g;
  if(x * x >= r * r) {
    if(!first) return (0,0);
    real d = 2 * sqrt(1-x*x);
    real start = -sqrt(1-x*x);
    for(int i = 0; i < 1000; ++i){
      g = g--(A((x,d*i/ITER + start)));
    }	
  }
  else {
    real d = sqrt(1-x*x) - sqrt(r*r-x*x);
    real start = sqrt(r*r-x*x);
    if(!first) start = -sqrt(1-x*x);
    for(int i = 0; i < 1000; ++i){
      g = g--(A((x,d*i/ITER + start)));
    }
  }
  path p = g;
  return p;
}
draw(circle((0,0),1));
draw(circle(0,r));
for(real t = -3*STEP; t <= 1; t += STEP){
	draw(Re(t, aut),red);
	draw(Re(t, aut, false),red);
	draw(Im(t, aut),blue);
	draw(Im(t, aut, false),blue);
}
xaxis("$\Re$",Ticks(Label(fontsize(8pt)),new real[]{}),xmin=-1.25,xmax=1.25);
yaxis("$\Im$",Ticks(Label(fontsize(8pt)),new real[]{}),ymin=-1.25,ymax=1.25);
\end{asy}
\hfill
\begin{asy}
int ITER = 1000;
real STEP = 0.26;
pair a = (0.6,-0.3);
real theta = 0.3pi;
real r = 0.4;

pair aut(pair X){
  return exp(I*theta) * r / X;
}

path Re(real y, pair A(pair), bool first = true){
  guide g;
  if(y * y >= r * r) {
    if(!first) return (0,0);
    real d = 2 * sqrt(1-y*y);
    real start = -sqrt(1-y*y);
    for(int i = 0; i < 1000; ++i){
      g = g--(A((d*i/ITER + start, y)));
    }	
  }
  else {
    real d = sqrt(1-y*y) - sqrt(r*r-y*y);
    real start = sqrt(r*r-y*y);
    if(!first) start = -sqrt(1-y*y);
    for(int i = 0; i < 1000; ++i){
      g = g--(A((d*i/ITER + start, y)));
    }
  }
  path p = g;
  return p;
}

path Im(real x, pair A(pair), bool first = true){
  guide g;
  if(x * x >= r * r) {
    if(!first) return (0,0);
    real d = 2 * sqrt(1-x*x);
    real start = -sqrt(1-x*x);
    for(int i = 0; i < 1000; ++i){
      g = g--(A((x,d*i/ITER + start)));
    }	
  }
  else {
    real d = sqrt(1-x*x) - sqrt(r*r-x*x);
    real start = sqrt(r*r-x*x);
    if(!first) start = -sqrt(1-x*x);
    for(int i = 0; i < 1000; ++i){
      g = g--(A((x,d*i/ITER + start)));
    }
  }
  path p = g;
  return p;
}
draw(circle((0,0),1));
draw(circle(0,r));
for(real t = -3*STEP; t <= 1; t += STEP){
	draw(Re(t, aut),red);
	draw(Re(t, aut, false),red);
	draw(Im(t, aut),blue);
	draw(Im(t, aut, false),blue);
}
xaxis("$\Re$",Ticks(Label(fontsize(8pt)),new real[]{}),xmin=-1.25,xmax=1.25);
yaxis("$\Im$",Ticks(Label(fontsize(8pt)),new real[]{}),ymin=-1.25,ymax=1.25);
\end{asy}
\caption{Primer avtomorfizma kolobarja}
\end{figure}


\subsection{Avtomorfizmi \texorpdfstring{$p$}{p}-povezanih območij}

Ta razdelek je prirejen po~\cite[poglavje 12]{krantz}
in~\cite[razdelek 5.9]{Artin_1991}.

Oglejmo si avtomorfizme območja
$\Omega = \rs \setminus \setb{x_i}{1 \leq i \leq p}$, pri čemer
brez škode za splošnost vzamemo $x_p = \infty$. Za $p = 1$ dobimo
kar kompleksno ravnino, katere avtomorfizme že poznamo. Pri $p = 2$
lahko brez škode za splošnost vzamemo $x_1 = 0$. Ni težko videti,
da so vsi avtomorfizmi oblike
$z \mapsto e^{i \theta} \cdot z^{\pm 1}$.

Sedaj si oglejmo še primer $p > 2$. Preverimo lahko, da se vsak
avtomorfizem $\Omega$ razširi do avtomorfizma Riemannove sfere, ki
permutira točke $x_i$. Ker je vsaka Möbiusova transformacija
enolično določena s tremi točkami, je moč grupe $\Aut(\Omega)$ tako
omejena s $p (p-1) (p-2)$. Izkaže se, da lahko to mejo še bistveno
izboljšamo. Najprej dokažimo naslednjo lemo:

\begin{lema}
Naj bo $p > 2$ naravno število in $f \in \Aut(\Omega)$ netrivialen
avtomorfizem območja
$\Omega = \rs \setminus \setb{x_i}{1 \leq i \leq p}$. Tedaj ima $f$
natanko dve fiksni točki.
\end{lema}

\begin{proof}
Naj bo
\[
f(z) = \frac{az + b}{cz + d}
\]
naš avtomorfizem. Ker je $\Aut(\Omega)$ končna grupa, obstaja tako
naravno število $n$, da je $f^n = \id$. Spomnimo se, da kompozitum
Möbiusovih transformacij ustreza množenju pripadajočih matrik.
Sedaj lahko poiščemo Jordanovo formo
\[
\begin{bmatrix}
a & b \\
c & d
\end{bmatrix}
=
P \cdot J \cdot P^{-1}
=
P \cdot
\begin{bmatrix}
\lambda & \varepsilon \\
   0    & \mu
\end{bmatrix}
\cdot P^{-1},
\]
pri čemer je $\varepsilon \in \set{0, 1}$. Če je $\varepsilon = 1$,
velja $\lambda = \mu$ in
\[
J^n =
\begin{bmatrix}
\lambda^n & n \cdot \lambda^{n-1} \\
    0     & \lambda^n
\end{bmatrix}
\ne I.
\]
Tako sledi $\varepsilon = 0$. Sledi, da se $f$ konjugira k
preslikavi
\[
z \mapsto \frac{\lambda z}{\mu} = \alpha z,
\]
ta pa ima natanko dve fiksni točki; $0$ in $\infty$.
\end{proof}

\begin{trditev}
Naj bo $p > 2$ naravno število in
$\Omega = \rs \setminus \setb{x_i}{1 \leq i \leq p}$. Tedaj velja
$\abs{\Aut(\Omega)} \leq 2p$ ali pa
$\abs{\Aut(\Omega)} \in \set{12, 24, 60}$.
\end{trditev}

\begin{proof}
Označimo $G = \Aut(\Omega)$ in predpostavimo, da je
$\abs{G} \geq 2$. Avtomorfizme območja $\Omega$ lahko razširimo do
avtomorfizmov $\rs$. Za vsako točko $z \in \rs$ njen stabilizator
označimo z $G_z = \setb{g \in G}{g(z) = z}$. Opazimo, da je točk z
netrivialnim stabilizatorjem končno mnogo -- v nasprotnem primeru
bi obstajal netrivialen avtomorfizem $T \in G$ z neskončno mnogo
fiksnimi točkami, kar bi pomenilo, da se na Riemannovi sferi v
neskončno mnogo točkah ujema z identiteto. Ker je Riemannova sfera
kompaktna, po principu identičnosti sledi $T = \id$.

Vzemimo neko maksimalno množico
$\setb{z_j \in \rs}{1 \leq j \leq r}$ točk, ki imajo netrivialen
stabilizator in paroma disjunktne orbite
$G \cdot z_j = \setb{g(z_j)}{g \in G}$. Naj bo še
$v_j = \abs{G_{z_j}}$ za vse $j \leq r$. Sedaj na dva načina
preštejmo fiksne točke netrivialnih avtomorfizmov (z
večkratnostmi). Ker ima vsak netrivialen avtomorfizem natanko $2$
fiksni točki, je teh skupaj enako kar $2 \cdot \br{\abs{G} - 1}$.
Vsaka fiksna točka netrivialnega avtomorfizma ima očitno
netrivialen stabilizator, zato je vsebovana v eni izmed orbit
$G \cdot z_j$. Velja tudi obratno, vsaka točka orbite $G \cdot z_j$
je fiksna točka natanko $v_j - 1$ netrivialnih avtomorfizmov, saj
so si stabilizatorji med seboj konjugirani. Elementov orbite je
seveda $\frac{\abs{G}}{v_j}$. Tako dobimo enakost
\[
2 \cdot \br{\abs{G} - 1} =
\sum_{j=1}^r \frac{\abs{G}}{v_j} \cdot \br{v_j - 1}.
\]
Z malo spretnosti lahko to enakost preoblikujemo v
\[
2 - \frac{2}{\abs{G}} = \sum_{j=1}^r \br{1 - \frac{1}{v_j}}.
\]
Sedaj ločimo naslednje primere:

\begin{enumerate}[i)]
\item Če je $r = 1$, velja
\[
1 - \frac{1}{v_1} < 1 \leq 2 - \frac{2}{\abs{G}},
\]
kar je seveda protislovje.

\item Naj bo $r = 2$. Enakost lahko v tem primeru prepišemo v
obliko
\[
\frac{2}{\abs{G}} = \frac{1}{v_1} + \frac{1}{v_2}.
\]
Ker velja $v_1, v_2 \leq \abs{G}$, je lahko zgornja enakost
izpolnjena le, kadar velja $v_1 = v_2 = \abs{G}$. To pomeni, da
imamo natanko dve točki, ki sta fiksni točki vsakega avtomorfizma.
Ker je vsak avtomorfizem Riemannove sfere natanko določen s tremi
točkami, je tako dovolj določiti, kam se slika ena izmed $p$ točk,
za to pa imamo kvečjemu $p$ možnosti. V tem primeru tako velja
$\abs{\Aut(\Omega)} \leq p$.\footnote{Dokazati se da celo, da velja
$\Aut(\Omega) \cong \Z_k$ za nek $k \leq p$.}

\item Naj bo $r = 3$. V tem primeru lahko zgornjo enakost prepišemo
v
\[
\frac{2}{\abs{G}} =
\frac{1}{v_1} + \frac{1}{v_2} + \frac{1}{v_3} - 1.
\]
Brez škode za splošnost naj velja $v_1 \leq v_2 \leq v_3$. Če za
vse $j$ velja $v_j \geq 3$, je desna stran enačbe nepozitivna. Tako
sledi $v_1 = 2$. Če velja tudi $v_2 = 2$, lahko izrazimo
$v_3 = \frac{\abs{G}}{2}$. Orbita točke $z_3$ je tako enaka kar
$\set{z_3, z_3'}$. Sledi, da stabilizatorji točke $z_3$ fiksirajo
ta dva elementa, kar po istem argumentu kot zgoraj implicira, da je
$v_3 \leq p$ in zato $\abs{G} \leq 2p$.\footnote{V tem primeru je
$\Aut(\Omega)$ izomorfna neki diedrski grupi.}

Ostane še primer, ko je $v_1 = 2$ in $v_2, v_3 \geq 3$. Če velja
$v_3 \geq 6$, dobimo
\[
\frac{1}{v_1} + \frac{1}{v_2} + \frac{1}{v_3} - 1 \leq
\frac{1}{2} + \frac{1}{3} + \frac{1}{6} - 1 = 0,
\]
kar je seveda protislovje. Podobno pridemo do protislovja, če velja
$v_2, v_3 \geq 4$. Tako nam preostanejo le še primeri
\[
(v_1, v_2, v_3) \in \set{(2,3,3), (2,3,4), (2,3,5)}.
\]
S krajšim računom dobimo $\abs{G} \in \set{12, 24, 60}$.

\item Naj bo $r \geq 4$. Ker je $v_j \geq 2$, sledi
\[
\sum_{j=1}^r \br{1 - \frac{1}{v_j}} \geq
\frac{r}{2} \geq 2 > 2 - \frac{2}{\abs{G}},
\]
kar je seveda protislovje. \qedhere
\end{enumerate}
\end{proof}

\begin{zgled}
Naj bo $p = 7$, $A = \set{0} \cup \setb{z \in \C}{z^6 = 1}$ in
$\Omega = \rs \setminus A$. Ker Möbiusove transformacije slikajo
krožnice v premice ali krožnice, se enotska krožnica preslika v
premico ali krožnico. Ta mora vsebovati $6$ točk iz množice $A$,
kar je mogoče le, če se slika sama vase. Sledi, da je $0$ fiksna
točka vsakega avtomorfizma tega območja. To med drugim pomeni, je
vsak avtomorfizem $T \in \Aut(\Omega)$ tudi avtomorfizem diska.
Tako so edini avtomorfizmi rotacije, ki pa fiksirajo tudi točko
$\infty$. To pomeni, da to območje spada v primer $r = 2$. Res
velja $\abs{\Aut(\Omega)} = 6 < p$.
\end{zgled}

\begin{zgled}
Naj bo $p = 7$, $A = \setb{z \in \C}{z^7 = 1}$ in
$\Omega = \rs \setminus A$. Podobno kot v prejšnjem zgledu
ugotovimo, da avtomorfizmi območja $\Omega$ slikajo enotsko
krožnico samo vase. Ker se poleg tega ohranja zaporedje točk na
krožnici, vsak avtomorfizem ustreza neki simetriji pravilnega
$7$-kotnika. Ker lahko rotacijo predstavimo z avtomorfizmom
$z \mapsto e^{\frac{2 \pi i}{7}} \cdot z$, zrcaljenje pa z
avtomorfizmom $z \mapsto \frac{1}{z}$, dobimo diedrsko grupo. Ta
zgled ustreza primeru, ko je $r = 3$ in $v_1 = v_2 = 2$. Res je
$\abs{\Aut(\Omega)} = 14 = 2p$.
\end{zgled}

\begin{zgled}
Naj bo $p = 4$, $A$ množica središč mejnih ploskev tetraedra,
očrtanega enotski sferi, in $\Omega = \rs \setminus A$. Območje
$\Omega$ lahko s stereografsko projekcijo predstavimo v kompleksni
ravnini, in sicer kot
$\Omega = \C \setminus \setb{z \in \C}{z^3 = 1}$. Naj bo
$\omega = e^{\frac{2 \pi i}{3}}$. Če je $\infty$ fiksna točka
avtomorfizma, je ta kar rotacija okoli izhodišča
$z \mapsto \omega^k \cdot z$, saj mora biti afina preslikava.
Preverimo lahko, da je tudi preslikava
\[
z \mapsto \frac{\omega z + 1}{z - \omega^2}
\]
avtomorfizem območja $\Omega$, ki prav tako ustreza rotaciji
tetraedra. Ti preslikavi generirata simetrično grupo tetraedra.
Tako sledi $\abs{\Aut(\Omega)} \geq 12$. Ker je $\Aut(\Omega)$
podgrupa permutacijske grupe, velja $\abs{\Aut(\Omega)} \mid 24$.
Ker pa po zgornjem premisleku $\Aut(\Omega)$ ne vsebuje preslikave,
ki bi fiksirala $1$ in $\infty$ ter hkrati zamenjala točki $\omega$
in $\omega^2$, sledi $\abs{\Aut(\Omega)} < 24$, oziroma
$\abs{\Aut(\Omega)} = 12$. Predstavnika orbit velikosti $4$ sta $0$
in $1$, orbite velikosti $6$ pa $1 + \sqrt{3}$.
\end{zgled}

Za primera $\abs{G} = 24$ in $\abs{G} = 60$ območje $\Omega$ dobimo
na enak način, le namesto tetraedra vzamemo oktaeder oziroma
ikozaeder.

Za katera števila $p$ pa lahko velja $\abs{G} > 2p$? Če imamo v
točki $x_j$ luknjo območja $\Omega$, jo moramo imeti tudi v vsakem
elementu njene orbite. Oglejmo si vsak primer posebej:

\begin{enumerate}[i)]
\item Naj bo $\abs{G} = 12$. V tem primeru imamo dve orbiti
velikosti $4$ in eno orbito velikosti $6$, vse ostale pa so
velikosti $12$. Tako sledi
\[
p = 4a + 6b + 12c,
\]
pri čemer veljajo omejitve $a \leq 2$ in $b \leq 1$. Ker želimo še
$p < 6$, je edina možnost kar $p = 4$.

\item Naj bo $\abs{G} = 24$. V tem primeru imamo po eno orbito
velikosti $6$, $8$ in $12$, zato velja
\[
p = 6a + 8b + 12c + d
\]
z omejitvami $a, b, c \leq 1$. Ob pogoju $p < 12$ tako sledi
$p \in \set{6, 8}$.

\item Naj bo $\abs{G} = 60$. V tem primeru imamo po eno orbito
velikosti $12$, $20$ in $30$, od koder sledi
\[
p = 12a + 20b + 30c + 60d
\]
z omejitvami $a, b, c \leq 1$. Z dodatnim pogojem $p < 30$ sta tako
edini možnosti $p \in \set{12, 20}$.
\end{enumerate}

Oglejmo si še primer, ko robne komponente niso nujno točke.

\begin{izrek}[Heins]
Naj bo $\Omega$ območje na Riemannovi sferi, katerega robne
komponente sestavlja $p$ točk ali Jordanovih krivulj. Tedaj za
moč grupe $\Aut(\Omega)$ veljajo enake ocene kot zgoraj.
\end{izrek}

\begin{proof}
Po posplošitvi Riemannovega upodobitvenega izreka
(glej~\cite{Koebe_uni}) lahko predpostavimo, da je $\Omega$ kar
enotski disk, ki mu odstranimo nekaj točk in diskov. Po
Carathéodoryjevem izreku sledi, da avtomorfizem permutira robne
komponente območja $\Omega$. Tako obstaja homomorfizem
$\lambda \colon \Aut(\Omega) \to S_{\partial \Omega}$, kjer je
$S_{\partial{\Omega}}$ permutacijska grupa. Jedro homomorfizma so
natanko tisti avtomorfizmi, ki fiksirajo vsako robno komponento
posebej. Pokažimo, da je jedro trivialno.

Naj bo $T \in \ker \lambda$ poljuben avtomorfizem. Najprej se lahko
znebimo točkastih robnih komponent, saj se te preslikajo same vase.
Z uporabo Schwarzovega zrcaljenja lahko $T$ razširimo preko vseh
ostalih robnih komponent. Enostavno je preveriti, da se pri tem
ohrani injektivnost preslikave. V limiti nam v vsaki komponenti
$\Omega^\mathsf{c}$ ostane samo ena točka. V vseh omejenih
komponentah je to odpravljiva singularnost, v neomejeni pa ali
odpravljiva singularnost ali pol prve stopnje. V prvem primeru je
$T$ omejen in zato konstanten, torej ne more biti avtomorfizem
območja $\Omega$. V drugem primeru se $T$ razširi do avtomorfizma
Riemannove sfere. Zanimajo nas torej avtomorfizmi Riemannove sfere,
ki fiksirajo vse robne komponente $\Omega$. V nadaljevanju na
točkaste robne komponente glejmo kot na diske z radijem $0$.

Naj bo $T = \frac{az + b}{cz + d}$. Sedaj obravnavajmo dva primera.

\begin{enumerate}[i)]
\item Velja $c=0$. V tem primeru je $T$ afina transformacija. To
implicira, da $T$ poleg robnih komponent fiksira tudi njihova
središča. Ker sta to vsaj dve različni točki, ima $T$ dve fiksni
točki, zato velja kar $T = \id$.

\item Velja $c \ne 0$. Tedaj je $T$ kompozitum inverzije v točki
$z_0 = -\frac{d}{c}$, rotacije in translacije. Ker mora inverzija
ohranjati radije robnih komponent (preostali preslikavi sta namreč
izometriji), morajo vse biti ortogonalne na neko krožnico s
središčem v $z_0$.\footnote{Ne morejo imeti središča v $z_0$, saj
bi tedaj $T$ območje $\Omega$ slikal v notranjost te krožnice.}
Sedaj opazimo, da smo prišli do protislovja, saj je inverzija
obrnila orientacijo središč robnih komponent, česar pa ne moremo
popraviti z rotacijo in translacijo.
\end{enumerate}

\begin{figure}[!ht]
\centering
\begin{asy}
path orth(pair X, string s){
	dot(s, X, dir(X));
	return shift(X) * scale(sqrt(abs(X)^2-1)) * unitcircle;
}

draw(circle(0,1), red+dashed);

path[] g;
g = g ^^ orth(1.8, "$0$");
g = g ^^ reverse(orth(1.1*dir(25), "$S_A$"));
dot("$S_B$", dir(-5), 2dir(-5));
g = g ^^ reverse(orth(1.08*dir(-32), "$S_C$"));
filldraw(g, cyan+opacity(0.2));
label("$\Omega$", shift(1.8) * scale(sqrt(1.8^2-1)) * dir(45), align = N+E);

dot("$z_0$", 0, dir(180));
\end{asy}
\qquad
\begin{asy}
path orth(pair X, string s){
	dot(s, X, dir(X));
	return shift(X) * scale(sqrt(abs(X)^2-1)) * unitcircle;
}

draw(circle(0,1), red+dashed);

path[] g;
g = g ^^ orth(1.8, "$0$");
g = g ^^ reverse(orth(1.1*dir(-25), "$S_A'$"));
dot("$S_B'$", dir(5), 2dir(5));
g = g ^^ reverse(orth(1.08*dir(32), "$S_C'$"));
filldraw(g, cyan+opacity(0.2));
label("$\Omega$", shift(1.8) * scale(sqrt(1.8^2-1)) * dir(45), align = N+E);

dot("$z_0$", 0, dir(180));
\end{asy}
\caption{Inverzija prezrcali robne komponente}
\end{figure}


Sedaj si izberimo poljubno točko $z \not \in \Omega$ in si oglejmo
njeno orbito, torej množico $A = \setb{T(z)}{T \in \Aut(\Omega)}$.
Po enakem razmisleku kot zgoraj se namreč vsak avtomorfizem
$\Omega$ razširi do avtomorfizma kompleksne ravnine. V vsaki
komponenti $\Omega^\mathsf{c}$ imamo kvečjemu en element množice
$A$. Sedaj si izberimo poljubno točko v eni izmed komponent, ki je
množica $A$ ne obišče, in postopek ponovimo. Tako na koncu dobimo
množico $S = \setb{z_i}{1 \leq i \leq p}$, pri čemer je v vsaki
komponenti $\Omega^\mathsf{c}$ natanko ena točka. Poljubni
avtomorfizem $T \in \Aut(\Omega)$ se torej razširi do avtomorfizma
območja $\Omega \setminus S$, s tem pa je izrek dokazan.
\end{proof}

Za konec še omenimo, da avtomorfizmov ni nujno končno mnogo, če
ima $\Omega^\mathsf{c}$ neskončno mnogo komponent. Na
sliki~\ref{sl:inf} je območje
\[
\Omega = \dsk \bigsetminus \bigcup_{n \in \Z} f^n(D),
\]
kjer je $D$ zaprt disk s središčem v $0$ in radijem $\frac{1}{5}$
ter
\[
f(z) = \frac{z - \frac{1}{2}}{1 - \frac{1}{2} z}.
\]

\begin{figure}[!ht]
\centering
\begin{asy}
real STEP = pi/10;
real R = 0.2;

pair aut(pair z) {
	return (z - 0.5) / (1 - 0.5 * z);
}

pair inv(pair z) {
	return (z + 0.5) / (1 + 0.5 * z);
}

pair power(pair z, int n) {
	if(n == 0) return z;
	if(n > 0) return power(aut(z), n-1);
	if(n < 0) return power(inv(z), n+1);
	return 0;
}

path iter(real r, int n) {
	guide p;
	for(real t = 0; t < 2pi; t += STEP) {
		p = p .. power(r * exp(-I * t), n);
	}
	return p .. cycle;
}

path[] g;
g = g ^^ circle(0,1);
for(int n = -10; n < 10; ++n) {
	g = g ^^ iter(R, n);
}

filldraw(g, cyan+opacity(0.2), linewidth(0));
label("$\Omega$", dir(45), align = N+E);
label("$D$", 0.2 * dir(45), align = S+W);
\end{asy}
\caption{Neskončno-povezano območje z neskončno avtomorfizmi}
\label{sl:inf}
\end{figure}


Seveda je $f^n \in \Aut(\Omega)$ za vsak $n \in \Z$. Ker sta lastni
vrednosti matrike
\[
\begin{bmatrix}
      1      & -\frac{1}{2} \\
-\frac{1}{2} &       1
\end{bmatrix}
\]
$\frac{1}{2}$ in $\frac{3}{2}$, je njena Jordanova forma enaka
\[
J =
\begin{bmatrix}
\frac{1}{2} &      0      \\
     0      & \frac{3}{2}
\end{bmatrix}.
\]
Tako je $J^n$ skalarni večkratnik identitete natanko tedaj, ko je
$n = 0$, kar pomeni, da $f$ nima končnega reda. Grupa
$\Aut(\Omega)$ je v tem primeru torej res neskončna.

\section{Riemannove ploskve}

\subsection{Gladke in kompleksne mnogoterosti}

%TODO Definicije gladkih mnogoterosti

\begin{definicija}
\emph{Meromorfen diferencial} Riemannove ploskve je dodelitev
meromorfne funkcije $f$ vsaki lokalni koordinati, pri čemer je
$f(z)\,dz$ neodvisna od lokalne koordinate.
\end{definicija}

Naj bosta $(U, \varphi)$ in $(V, \psi)$ lokalni karti, za kateri
velja $U \cap V \ne \emptyset$. Če jima meromorfen diferencial
$\omega$ priredi funkciji $\omega_U$ in $\omega_V$, mora tako
veljati $\omega_U = \omega_V \cdot \br{\psi \circ \varphi^{-1}}'$.

\begin{trditev}
Naj bosta $\alpha$ in $\beta$ meromorfna diferenciala. Tedaj je
$\frac{\alpha}{\beta}$ meromorfna funkcija.
\end{trditev}

\begin{proof}
Z zgornjimi oznakami velja
\[
\frac{\alpha_U}{\beta_U} =
\frac{\alpha_V \cdot \br{\psi \circ \varphi^{-1}}'}
{\beta_V \cdot \br{\psi \circ \varphi^{-1}}'} =
\frac{\alpha_V}{\beta_V}.
\]
Kvocient $\frac{\alpha}{\beta}$ tako ni odvisen od lokalnih
koordinat.
\end{proof}

Očitno velja tudi obratno -- če je $\alpha$ meromorfen diferencial
in $f$ meromorfna funkcija, je tudi $f \alpha$ meromorfen
diferencial.

\begin{trditev}
\label{td:deg}
Naj bo $f \colon M \to N$ nekonstantna holomorfna preslikava med
kompaktnima Riemannovima ploskvama. Tedaj obstaja naravno število
$m$, za katero $f$ doseže vsako točko $Q \in N$ natanko
$m$-krat.\footnote{Šteto z večkratnostmi.}
\end{trditev}

\begin{proof}
Iz kompleksne analize vemo, da za vsako točko $P \in M$ obstajajo
take lokalne koordinate $\tilde{z}$, da je
$f(\tilde{z}) = f(P) + \tilde{z}^n$. Število $n-1$ označimo z
$b(P)$ in mu pravimo BRANCHING NUMBER.

Za vsako naravno število $m$ naj bo
\[
\Sigma_m =
\setb{X \in N}{\sum_{f(P)=X} (b(P)+1) \geq m}.
\]
Označimo še
\[
\varphi(X) = \sum_{f(P) = X} (b(P)+1).
\]
Vse množice $\Sigma_m$ so odprte -- če je $b(P) = n-1$, lahko v
lokalnih koordinatah zapišemo $f(\tilde{z}) = \tilde{z}^n$. Enačba
$f(\tilde{z}) = \varepsilon$ ima tako natanko $n$ rešitev, zato za
okolico $U$ točke $P$ velja
\[
b(P) + 1 = \sum_{Q \in U \cap f^{-1}(P')} (b(Q) + 1),
\]
kjer je $P' \in f(U)$. Če to enakost seštejemo po okolicah vseh točk $P \in f^{-1}(X)$, dobimo
\[
m \leq \varphi(X) \leq \varphi(P').
\]

Pokažimo še, da so te množice zaprte v $\rs$. Naj bo $Q$ limita
zaporedja točk $Q_k \in \Sigma_m$, pri čemer je brez škode za
splošnost $b(P) = 0$ za vsak $P \in f^{-1}(Q_k)$. Ker imajo vse
množice $f^{-1}(Q_k)$ vsaj $m$ elementov, lahko najdemo tako
podzaporedje zaporedja $(Q_k)_{n=1}^\infty$, da lahko iz njihovih
praslik tvorimo $m$ konvergentnih zaporedij. Tako sledi
\[
\sum_{P \in f^{-1}(Q)} (b(P)+1) \geq m.
\]
Sledi, da so vse množice $\Sigma_m$ odprte in zaprte v $\rs$. Čim
je ena izmed množic $\Sigma_m$ neprazna, je tako enaka celotni
Riemannovi sferi, saj je ta povezana.
\end{proof}

Številu $m$ pravimo \emph{stopnja} preslikave $f$.

\begin{definicija}
Za kompaktni Riemannovo ploskvi $M$ in $N$ ter nekonstantno
preslikavo $f \colon M \to N$ definiramo
\emph{TOTAL BRANCHING NUMBER} kot
\[
B = \sum_{P \in M} b_f(P).
\]
\end{definicija}

Število je dobro definirano, saj je množica
$\setb{P \in M}{b_f(P)}$ diskretna in tako zaradi kompaktnosti
končna.

\begin{izrek}[Riemann-Hurwitz]
\label{iz:rie-hur}
Naj bosta $M$ in $N$ kompaktni Riemannovi ploskvi rodov $g$ in
$\gamma$, $f \colon M \to N$ pa nekonstantna preslikava stopnje
$n$. Tedaj za TOTAL BRANCHING NUMBER $B$ velja
\[
g = n(\gamma - 1) + 1 + \frac{B}{2}.
\]
\end{izrek}

\begin{proof}
Ker je množica $\setb{P \in M}{b_f(P) > 0}$ končna, jo lahko
uporabimo za triangulacijo ploskve $N$. Denimo, da ima
triangulacija $F$ lic, $E$ povezav in $V$ vozlišč. To
triangulacijo lahko z $f^{-1}$ preslikamo na $M$. Tako dobimo
triangulacijo ploskve $M$ z $nF$ lici, $nE$ povezavami in $nV - B$
vozlišči. Sledi, da je
\begin{align*}
F - E + V &= 2 - 2 \gamma,
\\
nF - nE + nV - B &= 2 - 2g.
\end{align*}
Iz teh enačb očitno sledi
\[
g = n(\gamma - 1) + 1 + \frac{B}{2}. \qedhere
\]
\end{proof}

%TODO Struktura na kvocientih
%\begin{definicija}
%Naj bo $H \subseteq \Aut M$ končna podgrupa grupe avtomorfizmov
%ploskve $M$. Na množici $\kvoc{M}{H}$ uvedemo kompleksno strukturo
%\end{definicija}

\subsection{Riemann-Rochov izrek}

\begin{definicija}
\emph{Delitelj} na Riemannovi ploskvi $M$ je formalni simbol
\[
\mathfrak{A} = \prod_{P \in M} P^{\alpha(P)},
\]
kjer za vsak $P$ velja $\alpha(P) \in \Z$ in je $\alpha(P) \ne 0$
za kvečjemu končno mnogo točk $P \in M$. \emph{Stopnja} delitelja
$\mathfrak{A}$ je definirana kot
\[
\deg \mathfrak{A} = \sum_{P \in M} \alpha(P).
\]
\end{definicija}

Delitelji na $M$ tvorijo grupo za naravno definirano množenje --
to grupo označimo z $\Div(M)$. Tako je
$\deg \colon \Div(M) \to \Z$ homomorfizem grup.

Za vsako neničelno meromorfno funkcijo $f \in \mathscr{K}(M)$
definiramo njen \emph{glavni delitelj} kot
\[
(f) = \prod_{P \in M} P^{\ord_P f}.
\]
Definiramo lahko še \emph{polarni delitelj}
\[
f^{-1}(\infty) = \prod_{P \in M} P^{\max(-\ord_P f, 0)}
\]
in \emph{ničelni delitelj}
\[
f^{-1}(0) = \prod_{P \in M} P^{\max(\ord_P f, 0)}.
\]
Opazimo, da velja
\[
(f) = \frac{f^{-1}(0)}{f^{-1}(\infty)}.
\]

\begin{lema}
Naj bo $M$ kompaktna Riemannova ploskev. Za vsako neničelno
funkcijo $f \in \mathscr{K}(M)$ velja
$\deg f^{-1}(0) = \deg f^{-1}(\infty)$. Ekvivalentno je
$\deg (f) = 0$.
\end{lema}

\begin{proof}
Stopnja polarnega delitelja funkcije $f$ je natanko število njenih
polov, štetih z večkratnostmi, stopnja ničelnega delitelja pa
število njenih ničel. Ti števili sta enaki po
trditvi~\ref{td:deg}.
\end{proof}

Posebej velja opomniti, da to pomeni, da imajo funkcije na
kompaktnih Riemannovih ploskvah enako število ničel in polov
(štetih z večkratnostmi).

Na deliteljih lahko uvedemo relacijo delne urejenosti kot
\[
\prod_{P \in M} P^{\alpha(P)} \geq \prod_{P \in M} P^{\beta(P)}
\iff
\forall P \in M \colon \alpha(P) \geq \beta(P).
\]
%TODO Definiraj še \sim in CELOST
Ni težko videti, da je za vsak delitelj $\mathfrak{A}$ na $M$
množica
\[
L(\mathfrak{A}) =
\setb{f \in \mathscr{K}(M)}{(f) \geq \mathfrak{A}}
\]
vektorski prostor -- njegovo dimenzijo označimo z
$r(\mathfrak{A})$.

\begin{zgled}
Velja $r(1) = 1$. Pogoj $(f) \geq 1$ je namreč ekvivalenten temu,
da je $f$ holomorfna. Ker so vse holomorfne funkcije na kompaktnih
Riemannovih ploskvah konstantne, je zato $L(1) \cong \C$, kar je
enodimenzionalen prostor.
\end{zgled}

\begin{zgled}
Če je $\deg \mathfrak{A} > 0$, je $r(\mathfrak{A}) = 0$. Iz
neenakosti $(f) \geq \mathfrak{A}$ za neničenlno funkcijo $f$
namreč sledi $0 = \deg (f) \geq \deg \mathfrak{A} > 0$, kar je
protislovje.
\end{zgled}

%TODO Nekje omeni zakaj so holomorfne konstantne

Podobno je tudi
\[
\Omega(\mathfrak{A}) =
\setb{\omega}{\text{$\omega$ je meromorfna $1$-forma} \land
(\omega) \geq \mathfrak{A}}
\]
vektorski prostor. Označimo
$i(\mathfrak{A}) = \dim \Omega(\mathfrak{A})$.

%TODO Definiraj še (\omega)

\begin{trditev}\label{td:mero_dif}
Naj bo $\mathfrak{A}$ poljuben delitelj in $\omega$ meromorfen
diferencial. Tedaj je
\[
i(\mathfrak{A}) = r \br{\mathfrak{A} (\omega)^{-1}}.
\]
\end{trditev}

\begin{proof}
Naj bo $\varphi \colon
\Omega(\mathfrak{A}) \to L \br{\mathfrak{A} (\omega)^{-1}}$
preslikava s predpisom
$\varphi \colon \zeta \mapsto \frac{\zeta}{\omega}$. Seveda je
predpis dobro definiran, ni pa težko videti, da je to izomorfizem
vektorskih prostorov. Sledi, da imata enako dimenzijo.
\end{proof}

\begin{izrek}[Riemann-Roch]
Naj bo $M$ kompaktna Riemannova ploskev roda $g$ in $\mathfrak{A}$
delitelj na $M$. Tedaj velja
\[
r \br{\mathfrak{A}^{-1}} =
\deg \mathfrak{A} - g + 1 + i(\mathfrak{A}).
\]
\end{izrek}

Dokaz izreka najdemo v~\cite[theorem~III.4.11]{Farkas_Kra_1980}.

\begin{zgled}
Z uporabo zgornjega izreka lahko izračunamo $i(1)$. Velja namreč
\[
i(1) = r(1) - \deg 1 + g - 1 = g. \qedhere
\]
\end{zgled}

\begin{trditev}
Naj bo $\deg \mathfrak{A} > 2g - 2$. Tedaj je
$i(\mathfrak{A}) = 0$.
\end{trditev}

\begin{proof}
Naj bo $\omega \in i(1)$ neničelna holomorfna $1$-forma. Tedaj je
\[
r \br{(\omega)^{-1}} = \deg (\omega) - g + 1 + i \br{(\omega)}.
\]
Po trditvi~\ref{td:mero_dif} je $r \br{(\omega)^{-1}} = i(1) = g$
in $i \br{(\omega)} = r(1) = 1$. Od tod sledi, da je
$\deg (\omega) = 2g - 2$.

Sedaj dobimo
\[
i(\mathfrak{A}) = r \br{\mathfrak{A} (\omega)^{-1}} = 0,
\]
saj je $\deg \br{\mathfrak{A} (\omega)^{-1}} > 0$.
\end{proof}

\subsection{Weierstrassove točke}

\begin{izrek}[Weierstrass]
Naj bo $M$ ploskev roda $g > 0$ in $P \in M$. Tedaj obstaja
natanko $g$ števil
\[
1 = n_1 < n_2 < \dots < n_g < 2g,
\]
za katera ne obstaja funkcija $f \in \mathscr{K}(M)$, ki je
holomorfna na $M \setminus \set{P}$ in ima pol reda $n_j$ v $P$.
Tem številom pravimo GAP.
\end{izrek}

\begin{proof}
Najprej opazimo, da je število $n$ GAP natanko tedaj, ko je
$r(P^{-n}) = r(P^{1-n})$. Ker je $r(P^{-n}) \leq r(P^{1-n}) + 1$,
število $n$ ni GAP natanko tedaj, ko velja
\[
r \br{P^{-n}} - r \br{P^{1-n}} = 1.
\]
Po Riemann-Rochovem izreku velja
\[
r \br{P^{-k}} = k-g+1 + i \br{P^k},
\]
zato sledi
\begin{align*}
r \br{P^{-n}} - r(1) &=
\sum_{k=1}^n \br{r \br{P^{-k}} - r \br{P^{1-k}}}
\\
&=
\sum_{k=1}^n \br{1 + i \br{P^k} - i \br{P^{k-1}}}
\\
&=
n + i \br{P^n} - i(1).
\end{align*}
Ker je $i(1) = g$ in za vse $n > 2g-2$ velja $i(P^n) = 0$, sledi
\[
r \br{P^{-n}} - 1 = n - g.
\]
Opazimo, da leva stran šteje ravno število NEGAPOV $\leq n$.
Sledi, da je GAPOV natanko $g$ in so vsi strogo manjši od $2g$.
\end{proof}

%TODO Posploši in dokaži lemo z determinanto Wronskega
\begin{definicija}
TEŽA točke $P \in M$ je vsota
\[
\tau(P) = \sum_{j=1}^g (n_j - j),
\]
kjer so $n_j$ GAPI za $P$.
\end{definicija}

\begin{lema}
Naj bo $M$ kompaktna Riemannova ploskev z rodom $g \geq 2$. Tedaj
je
\[
\sum_{P \in M} \tau(P) = g^3-g.
\]
\end{lema}

%TODO Proof: det Wronskega je m-ta potenca holomorfne diferencialne
%            forme, zato je vsota redov m * (2g-2)

\begin{definicija}
Točka $P \in M$ je \emph{Weierstrassova točka}, če na $M$ obstaja
neničelna holomorfna diferencialna $1$-forma z ničlo reda vsaj
$g$ v $P$.\footnote{V splošnem definiramo $q$-Weierstrassove točke
-- obstaja $q$-forma z ničlo reda vsaj $\dim \mathscr{H}^q(M)$.}
\end{definicija}

\begin{lema}
Ekvivalentno, vsaj eno izmed števil $2, \dots, g$ ni GAP.
\end{lema}

\begin{proof}
Obstoj diferencialne $1$-forme z ničlo reda vsaj $g$ v $P$ je
ekvivalentna pogoju $i(P^g) > 0$. Po Riemann-Rochovem izreku je ta
neenakost ekvivalentna
\[
r \br{P^{-g}} - 1 > 0,
\]
oziroma $r \br{P^{-g}} \geq 2$. Ker je $r(1) = 1$, med
$2, \dots, g$ obstaja število, ki ni GAP.
\end{proof}

\begin{lema}
Naj bo $M$ kompaktna Riemannova ploskev roda $g \geq 2$. Tedaj za
število $w$ Weierstrassovih točk veljata oceni
\[
2g + 2 \leq w \leq g^3 - g.
\]
\end{lema}

\begin{proof}
Ker je $\tau(P) \geq 1$ za vsako Weierstrassovo točko in velja
\[
\sum_{P \in M} = g^3 - g,
\]
takoj sledi $w \leq g^3 - g$. Velja pa
\begin{align*}
\tau(P) &= \sum_{j=1}^g (n_j - j)
\\
&=
\sum_{j=1}^{2g} j - \sum_{j=1}^g \alpha_j - \sum_{j=1}^g j
\\
&=
\sum_{j=g+1}^{2g-1} j - \sum_{j=1}^{g-1} \alpha_j
\\
&\leq \frac{g^2-g}{2}. \qedhere
\end{align*}
\end{proof}

%TODO Definicija \alpha_i in lemice

\subsection{Hipereliptične ploskve}

\begin{definicija}
Kompaktna Riemannova ploskev $M$ je \emph{hipereliptična}, če
obstaja nekonstantna meromorfna funkcija $f \colon M \to \rs$ z
natanko dvema poloma.\footnote{Pri tem pole štejemo z
večkratnostmi.}
\end{definicija}

Ekvivalentno to pomeni, da obstaja tak CEL delitelj
$D \in \Div M$, da je $\deg D = 2$ in $r(D^{-1}) \geq 2$.

\begin{trditev}
Vsaka kompaktna Riemannova ploskev roda $g \leq 2$ je
hipereliptična.
\end{trditev}

\begin{trditev}
Weierstrassove ploskve imajo natanko $2g+2$ BRANCH točk.
\end{trditev}

\begin{proof}
Po izreku~\ref{iz:rie-hur} velja
\[
g = 2 \cdot (0-1) + 1 + \frac{B}{2}. \qedhere
\]
\end{proof}

\begin{trditev}
BRANCH točke preslikave $f$ so natanko Weierstrassove točke ploskve
$M$.
\end{trditev}

\begin{proof}
Naj bo $P \in M$ BRANCH točka. Če je $P$ pol funkcije $f$, je
njegova stopnja tako enaka $2$. V nasprotnem primeru ima funkcija
\[
g \equiv \frac{1}{f - f(P)}
\]
pol stopnje $2$ v $P$. V obeh primerih sledi, da $2$ ni GAP za
točko $P$, zato je ta Weierstrassova.

Vsaka BRANCH točka ima tako TEŽO
\[
\sum_{k=1}^g \br{2k-1} - \sum_{k=1}^g k = \frac{1}{2} g (g-1),
\]
zato je njihova skupna teža $g^3 - g$. Sledi, da so to vse
Weierstrassove točke.
\end{proof}

\begin{lema}
Naj bo $P$ Weierstrassova točka hipereliptične ploskve $M$ in
$f \in \mathscr{K}(M)$ funkcija stopnje $2$. Tedaj velja
$f^{-1}(\infty) \sim P^2$.
\end{lema}

\begin{proof}
Točka $P$ je BRANCH točka funkcije $f$. Če je $P$ pol te funkcije,
je zato reda $2$ in je $f^{-1}(\infty) = P^2$. V nasprotnem primeru
definiramo funkcijo
\[
g = \frac{1}{f - f(P)}.
\]
Ni težko videti, da je $(g) = f^{-1}(\infty) P^{-2}$.
\end{proof}

\begin{trditev}
Naj bosta $f$ in $g$ dve funkciji $f \colon M \to \rs$ stopnje
$2$. Tedaj obstaja Möbiusova transformacija $A$, za katero je
\[
g = A \circ f.
\]
\end{trditev}

\begin{proof}
Naj bo $f^{-1}(\infty) = P_1 Q_1$ in $g^{-1}(\infty) = P_2 Q_2$.
Ker na $M$ ne obstajajo funkcije stopnje $1$, sledi
$r(P_1^{-1} Q_1^{-1}) = r(P_2^{-1} Q_2^{-1}) = 2$. Prostora
$L(P_1^{-1} Q_1^{-1})$ in $L(P_2^{-1} Q_2^{-1})$ imata tako
zaporedoma bazi $\set{1, f}$ in $\set{1, g}$. Ker za Weierstrassovo
točko $P$ velja $P_1 Q_1 \sim P^2 \sim P_2 Q_2$, sledi, da obstaja
meromorfna preslikava $h$, za katero je
$(h) = P_1 Q_1 P_2^{-1} Q_2^{-1}$. Ker je s predpisom
$\varphi \mapsto h \cdot \varphi$ očitno podan izomorfizem
prostorov $L(P_1^{-1} Q_1^{-1})$ in $L(P_2^{-1} Q_2^{-1})$,
obstajajo konstante $\alpha, \beta, \gamma$ in $\delta$, za katere
je
\[
1 = \alpha h + \beta hf
\quad \text{in} \quad
g = \gamma h + \delta hf.
\]
Tako lahko izrazimo
\[
g = \frac{\gamma + \delta f}{\alpha + \beta f}. \qedhere
\]
\end{proof}

\begin{trditev}
Naj bo $M$ kompaktna Riemannova ploskev roda $g$. Tedaj je $M$
hipereliptična natanko tedaj, ko obstaja involucija $J \in \Aut M$
z natanko $2g + 2$ fiksnimi točkami.
\end{trditev}

\begin{proof}
Predpostavimo najprej, da je $M$ hipereliptična. Naj bo
$f \colon M \to \rs$ meromorfna funkcija stopnje $2$. Za vsak
$P \in M$ tako obstaja še ena točka $Q \in M$, za katero je
$f(P) = f(Q)$ (če je $\ord_P f = 2$, vzamemo $Q=P$). Tako lahko
enostavno definiramo $J(P) = Q$. Ni težko videti, da je $J$
res involucija z $2g + 2$ fiksnimi točkami.

Če je $Q = J(P) \ne P$, lahko na okolici $U_Q$ točke $Q$ zapišemo
\[
J(X) = \br{\eval{f}{U_Q}{}}^{-1}(f(X)),
\]
zato je $J$ holomorfna na $M \setminus W$. Če pa je $J(P) = P$, pa
je $h = \sqrt{f - f(P)}$ lokalna koordinata, za katero velja
$J(h) = -h$, saj je
\[
f(P_h) = h^2 + f(P) = (-h)^2 + f(P) = f(P_{-h}).
\]
Tako je $J$ holomorfna tudi na $W$.

Predpostavimo sedaj, da obstaja involucija $J \in \Aut M$ z
$2g + 2$ fiksnimi točkami. Ker se projekcija
$f \colon M \to \kvoc{M}{\skl{J}}$ BRANCHA v natanko $2g + 2$
točkah, po izreku~\ref{iz:rie-hur} sledi, da je rod ploskve
$\kvoc{M}{\skl{J}}$ enak $0$. Sledi, da je
$\kvoc{M}{\skl{J}} \cong \rs$, zato je $f$ meromorfna funkcija z
dvema poloma.
\end{proof}

%TODO Zakaj je ta kvocient ploskev?

Opazimo, da so fiksne točke hipereliptične involucije natanko
Weierstrassove točke.

\begin{trditev}
Naj bo $M$ hipereliptična Riemannova ploskev roda $g \geq 2$ in
$T \in \Aut M$. Če je $T \not \in \skl{J}$, ima $T$ kvečjemu $4$
fiksne točke.
\end{trditev}

\begin{proof}
Naj bo $f \colon M \to \rs$ funkcija z natanko dvema poloma. Tedaj
je taka tudi $f \circ T$, zato obstaja Möbiusova transformacija
$A$, za katero je
\[
f \circ T = A \circ f.
\]
Naj bo $P$ fiksna točka avtomorfizma $T$. Sledi, da je
\[
A(f(P)) = f(T(P)) = f(P),
\]
zato je $f(P)$ fiksna točka preslikave $A$. Opazimo, da je
$A \ne \id$, saj bi v nasprotnem primeru veljalo $f \circ T = f$,
kar implicira $T \in \skl{J}$. Tako ima $A$ kvečjemu $2$ fiksni
točki, zato jih ima $T$ največ $4$.
\end{proof}

\section{Avtomorfizmi Riemannovih poloskev}

\subsection{Sfere in torusi}

Za določanje grupe avtomorfizmov Riemannovih ploskev so pomembne
njihove topološke lastnosti -- vsak avtomorfizem je namreč tudi
homeomorfizem. Iz geometrijske topologije vemo, da je vsaka
orientabilna kompaktna ploskev homeomorfna vsoti $g$ torusov.
Številu $g$ pravimo rod ploskve.

Najprej si oglejmo ploskve z ničelnim rodom -- topološko so to kar
sfere. V prejšnjih razdelkih smo ugotovili, da je grupa
avtomorfizmov Riemannove sfere enaka
\[
\Aut \br{\rs} = \setb{\frac{az + b}{cz + d}}{ad - bc = 1}.
\]

Vemo pa, da je grupa avtomorfizmov odvisna ne samo od topoloških
lastnosti objekta, ampak tudi njegove kompleksne strukture.
%TODO To ni problem

Naslednji izziv so ploskve z rodom $g=1$ -- torusi. Za toruse
IZREK ne velja, zato imamo več različnih grup avtomorfizmov.
Oglejmo si, kako jih dobimo:
%TODO Kvocienti in stvari

\subsection{Ploskve večjih rodov}

\begin{trditev}
Naj bo $T \in \Aut M$ netrivialen avtomorfizem. Tedaj ima $T$
največ $2g + 2$ fiksnih točk.
\end{trditev}

\begin{proof}
Naj bo $P \in M$ točka, za katero je $T(P) \ne P$. Tedaj obstaja
meromorfna funkcija $f \in \mathscr{K}(M)$ z divizorjem polov
$P^r$ za nek $1 \leq r \leq g + 1$. Oglejmo si funkcijo
$h = f - f \circ T$. Njen divizor polov je očitno
$P^r (T^{-1}P)^r$. Velja torej
\[
\deg h^{-1}(0) = \deg h^{-1}(\infty) = 2r \leq 2g + 2,
\]
zato ima $g$ kvečjemu $2g + 2$ ničel. Ni težko videti, da so njene
ničle natanko fiksne točke avtomorfizma $T$.
\end{proof}

\begin{lema}
Naj bo $M$ kompaktna Riemannova ploskev roda $g \geq 2$, $W$ pa
množica njenih Weierstrassovih točk. Tedaj ta vsak avtomorfizem
$T \in \Aut M$ velja $T(W) = W$.
\end{lema}

\begin{proof}
Avtomorfizmi ohranjajo GAPE.
\end{proof}

\begin{izrek}[Schwarz]
Grupe avtomorfizmov kompaktnih ploskev roda $g \geq 2$ so končne.
\end{izrek}

\begin{proof}
Po zgornji lemi sledi, da obstaja homomorfizem
$\lambda \colon \Aut M \to S_W$, kjer je $S_W$ simetrična grupa.
Dovolj je pokazati, da ima $\lambda$ končno jedro. Ločimo dva
primera.

\begin{enumerate}[a)]
\item Če $M$ ni hipereliptična, ima več kot $2g + 2$
Weierstrassovih točk. Vsak avtomorfizem, ki fiksira Weierstrassove
točke, je zato kar identiteta, zato je $\ker \lambda$ trivialno.

\item Če je $M$ hipereliptična, velja kar
$\ker \lambda = \skl{J}$, kjer je $J$ hipereliptična involucija,
velja pa $\abs{\skl{J}} = 2$. \qedhere
\end{enumerate}
\end{proof}

%TODO Hurwitz

% \section{Zaključek}
% ...

\end{document}
